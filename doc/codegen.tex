\documentclass[a4paper,fleqn]{article}

\usepackage{fullpage}

\usepackage{amsmath}
\usepackage{amssymb}
\usepackage{stmaryrd} % llbracket, rrbracket
%\usepackage{mathtools} % dcases
\usepackage{color}
\usepackage{listings}
\usepackage{hyperref}
\usepackage{microtype}
\usepackage{balance}
\usepackage{mathtools}

\definecolor{myviolet}{rgb}{0.6,0.0,0.65}
\definecolor{myblue}{rgb}{0.1,0.0,0.8}
\definecolor{mygreen}{rgb}{0.1,0.5,0.0}
\definecolor{myred}{rgb}{0.8,0.0,0.0}

\lstdefinelanguage{coq}{
  keywords=[1]{Check,Section,Definition,Defined,CoInductive,Coercion,Inductive,Extraction,Fixpoint,Parameter,Module,Import,Record,Structure,Axiom,Lemma,Theorem,Notation,Reserved,End,Proof,Goal,Qed,From,Require,Variable,Variables,Hypothesis,Let,Eval},
  keywordstyle=\color{myviolet}\ttfamily,
  morekeywords=[2]{match,with,end,Set,Prop,Type,fun,of,let,in,struct,if,is,then,else,as,return,fix,for,leta,letr,letn,letd,letp,nmatch,dmatch,app,rapp},
  keywordstyle=[2]\color{mygreen}\ttfamily,
  morekeywords=[3]{reflexivity},
  keywordstyle=[3]\color{myred}\ttfamily,
  morekeywords=[4]{CodeGen,Global,Local,Inline,Monadify,Action,Monomorphization,Return,Bind,Pure,Monadification,Terminate,GenC,Reset},
  keywordstyle=[4]\color{myblue}\ttfamily,
}

% The color of "in" is myviolet for
% Definition c := Eval ... in ...
\lstdefinelanguage{vernacular}{
  keywords=[1]{Check,Section,Definition,Defined,CoInductive,Coercion,Inductive,Extraction,Fixpoint,Parameter,Module,Import,Record,Structure,Axiom,Lemma,Theorem,Notation,Reserved,End,Proof,Goal,Qed,From,Require,Variable,Variables,Hypothesis,Let,Eval,in},
  keywordstyle=\color{myviolet}\ttfamily,
  morekeywords=[2]{match,with,end,Set,Prop,Type,fun,of,let,struct,if,is,then,else,as,return,fix,for,leta,letr,letn,letd,letp,nmatch,dmatch,app,rapp},
  keywordstyle=[2]\color{mygreen}\ttfamily,
  morekeywords=[3]{reflexivity},
  keywordstyle=[3]\color{myred}\ttfamily,
  morekeywords=[4]{CodeGen,Global,Local,Inline,Monadify,Action,Monomorphization,Return,Bind,Pure,Monadification,Terminate,GenC,Reset},
  keywordstyle=[4]\color{myblue}\ttfamily,
}

\lstdefinestyle{Cstyle} {language=C,basicstyle=\small\ttfamily}
\lstdefinestyle{OCamlstyle} {language=Caml,basicstyle=\small\ttfamily}
\lstdefinestyle{Vernacularstyle} {language=vernacular,basicstyle=\small\ttfamily}

\def\pipe{\char`\|}
\def\tilde{\char`\~}
\def\tiret{\char`\-}
\def\plus{\char`\+}
\def\myhat{\char`\^}
\def\mystar{\char`\*}
\def\mybs{\char`\\}
\def\placeholder{\char`\_}

\lstset{
  language=coq,
  columns=fullflexible,
  basicstyle=\small\ttfamily,
  identifierstyle=\color{black}\ttfamily,
  commentstyle=\color{myred}\ttfamily,
  morecomment=[n]{(*}{*)},
  morestring=[b][\ttfamily]",
  showstringspaces=false,
  keepspaces,
  literate=
  {->}{$\to$}1
  {forall}{$\forall$}1
  {\\}{\texttt{\mybs}}1
  {>>=}{{$\gg =$}}3
  {=>}{$\Rightarrow$}3
}

\def\coq{\textrm{Coq}}
\def\gallina{\textrm{Gallina}}
\def\ocaml{\textrm{OCaml}}
\def\haskell{\textrm{Haskell}}
\def\scheme{\textrm{Scheme}}
\def\ssreflect{\textrm{SSReflect}}
\def\oeuf{\textrm{\OE uf}}
\def\certicoq{\textrm{CertiCoq}}
\def\codegen{\textrm{Codegen}}

\newlength{\bnforlen}
\settowidth{\bnforlen}{$=$}
\newcommand{\bnfor}{\mathrel{\makebox[\bnforlen]{$|$}}}

\newcommand{\kwDefinition}{\mbox{\color{myviolet}\ttfamily Definition}}
\newcommand{\kwSection}{\mbox{\color{myviolet}\ttfamily Section}}

\newcommand{\kwlet}{\mbox{\color{mygreen}\ttfamily let}}
\newcommand{\kwin}{\mbox{\color{mygreen}\ttfamily in}}
\newcommand{\kwmatch}{\mbox{\color{mygreen}\ttfamily match}}
\newcommand{\kwas}{\mbox{\color{mygreen}\ttfamily as}}
\newcommand{\kwreturn}{\mbox{\color{mygreen}\ttfamily return}}
\newcommand{\kwwith}{\mbox{\color{mygreen}\ttfamily with}}
\newcommand{\kwend}{\mbox{\color{mygreen}\ttfamily end}}
\newcommand{\kwfix}{\mbox{\color{mygreen}\ttfamily fix}}
\newcommand{\kwfor}{\mbox{\color{mygreen}\ttfamily for}}

\newcommand{\lam}[2]{\lambda #1.\:#2}
\newcommand{\lamT}[3]{\lambda #1\mathord{:}#2.\:#3}
\newcommand{\lamB}[1]{\lambda #1.\:}
\newcommand{\lamTB}[2]{\lambda #1\mathord{:}#2.\:}
\newcommand{\lamM}[3]{\lambda \rep{#1}.\:#3}
\newcommand{\lamTM}[3]{\lambda \rep{#1\mathord{:}#2}.\:#3}

\newcommand{\gassum}[2]{(#1\mathord{:}#2)}
\newcommand{\glodef}[3]{(#1:=#2\mathord{:}#3)}
\newcommand{\glodefB}[2]{(#1:=#2)}
\newcommand{\lassum}[2]{(#1\mathord{:}#2)}
\newcommand{\ldef}[3]{(#1:=#2\mathord{:}#3)}

\newcommand{\prodT}[3]{\forall #1\mathord{:}#2.\:#3}

\newcommand{\letin}[3]{\kwlet\:#1:=#2\:\kwin\:#3}
\newcommand{\letinB}[2]{\kwlet\:#1:=#2\:\kwin}
\newcommand{\letinM}[3]{\kwlet\:\rep{#1:=#2}\:\kwin\:#3}

\newcommand{\omatch}[2]{\kwmatch\:#1\:\kwwith\:{#2}\:\kwend}
\newcommand{\match}[4]{\kwmatch\:#1\:\kwwith\:(#2 \Rightarrow #3)_{#4}\:\kwend}
\newcommand{\matchasinret}[7]{\kwmatch\:#1\:\kwas\:#2\:\kwin\:#3\:\kwreturn\:#4\:\kwwith\:(#5 \Rightarrow #6)_{#7}\:\kwend}

\newcommand{\ofix}[2]{\kwfix\:{#1}\:\kwfor\:{#2}}
\newcommand{\fix}[4]{\kwfix\:(#1 := #2)_{#3}\:\kwfor\:#4}
\newcommand{\fixT}[5]{\kwfix\:(#1:\!#2 := #3)_{#4}\:\kwfor\:#5}

\DeclareMathOperator{\NA}{NA} % number of arguments
\DeclareMathOperator{\NP}{NP} % number of parameters for the inductive type
\DeclareMathOperator{\NI}{NI} % number of indexes for the inductive type
\DeclareMathOperator{\NM}{NM} % number of members for the constructor

\DeclareMathOperator{\Arr}{Arr} % Arr(t) is arity of the inductive type t.
\newcommand{\arr}[1]{\Gamma_{\Arr(#1)}}

\DeclareMathOperator{\FV}{FV}
\DeclareMathOperator{\LC}{LC}
\DeclareMathOperator{\KV}{KV}
\DeclareMathOperator{\FIXFUNCS}{FIXFUNCS}
\DeclareMathOperator{\FIXK}{FIXK}
\DeclareMathOperator{\FIXFV}{FIXFV}
\DeclareMathOperator{\EXARGS}{EXARGS}

\newcommand{\tD}{{t^\mathrm{D}}}
\newcommand{\tE}{{t^\mathrm{E}}}
\newcommand{\tL}{{t^\mathrm{L}}}
\newcommand{\tM}{{t^\mathrm{M}}}
\newcommand{\tF}{{t^\mathrm{F}}}
\newcommand{\tC}{{t^\mathrm{C}}}
\newcommand{\tA}{{t^\mathrm{A}}}

\DeclareMathOperator{\merg}{merge}

\newcommand{\tbigcap}{{\textstyle\bigcap}}
\newcommand{\tbigcup}{{\textstyle\bigcup}}
\newcommand{\tbigvee}{{\textstyle\bigvee}}
\newcommand{\tbigwedge}{{\textstyle\bigwedge}}
\newcommand{\breakrule}{\\[0.5em]}

\newcommand{\BRA}[1]{\llbracket #1 \rrbracket}

\DeclareMathOperator{\genbody}{GENBODY}
\newcommand{\genbodyat}[2]{\genbody^\mathrm{AT}_{#1}\BRA{#2}}
\newcommand{\genbodyan}[1]{\genbody^\mathrm{AN}\BRA{#1}}
\newcommand{\genbodyb}[2]{\genbody^\mathrm{B}_{#1}\BRA{#2}}

\DeclareMathOperator{\genfunop}{GENFUN}
\newcommand{\genfun}[1]{\genfunop\BRA{#1}}
\newcommand{\genfuns}[1]{\genfunop^\mathrm{S}\BRA{#1}}
\newcommand{\genfunm}[1]{\genfunop^\mathrm{M}\BRA{#1}}

\newcommand{\enumentries}[1]{\mathrm{enum\_entries}\BRA{#1}}
\newcommand{\argstructdefs}[1]{\mathrm{arg\_structdefs}\BRA{#1}}
\newcommand{\mainstructdef}[1]{\mathrm{main\_structdef}\BRA{#1}}
\newcommand{\auxstructdef}[2]{\mathrm{aux\_structdef}\BRA{#1}_{#2}}
\newcommand{\forwarddecl}[1]{\mathrm{forward\_decl}\BRA{#1}}
\newcommand{\entryfunctions}[1]{\mathrm{entry\_functions}\BRA{#1}}
\newcommand{\mainfunction}[1]{\mathrm{main\_function}\BRA{#1}}
\newcommand{\auxfunction}[2]{\mathrm{aux\_function}\BRA{#1}_{#2}}
\newcommand{\bodyfunction}[1]{\mathrm{body\_function}\BRA{#1}}
\newcommand{\auxcase}[2]{\mathrm{aux\_case}\BRA{#1}_{#2}}
\newcommand{\maincase}[1]{\mathrm{main\_case}\BRA{#1}}

\DeclareMathOperator{\cvop}{CV}
\newcommand{\cv}[3]{\cvop\BRA{#1\,/\,#2}_{#3}}

\DeclareMathOperator{\TRop}{TR}
\newcommand{\TR}[1]{\TRop\BRA{#1}}
\newcommand{\tr}{\mathit{TR}}

\DeclareMathOperator{\RNTop}{RNT}
\newcommand{\RNT}[1]{\RNTop\BRA{#1}}

\DeclareMathOperator{\Nop}{N}
\DeclareMathOperator{\Top}{T}

\newcommand{\N}[1]{\Nop\BRA{#1}}
\newcommand{\T}[1]{\Top\BRA{#1}}

\DeclareMathOperator{\APP}{APP}
\DeclareMathOperator{\components}{components}

\newcommand{\ldq}{\text{``}}
\newcommand{\rdq}{\text{''}}
\newcommand{\dq}[1]{\text{``}#1\text{''}}
\newcommand{\ttlparen}{\texttt{(}}
\newcommand{\ttrparen}{\texttt{)}}
\newcommand{\ttparen}[1]{\texttt{(}#1\texttt{)}}
\newcommand{\ttlbrace}{\texttt{\char '173}}
\newcommand{\ttrbrace}{\texttt{\char '175}}
\newcommand{\ttbrace}[1]{\ttlbrace#1\ttrbrace}
\newcommand{\tteq}{\texttt{=}}
\newcommand{\ttsemi}{\texttt{;}}
\newcommand{\ttcomma}{\texttt{,}}
\newcommand{\ttcolon}{\texttt{:}}
\newcommand{\ttstar}{\texttt{*}}
\newcommand{\ttamp}{\texttt{\&}}

\newcommand{\kwswitch}{\mbox{\color{myviolet}\ttfamily switch}}
\newcommand{\kwbreak}{\mbox{\color{myviolet}\ttfamily break}}
\newcommand{\kwgoto}{\mbox{\color{myviolet}\ttfamily goto}}
\newcommand{\kwstruct}{\mbox{\color{myviolet}\ttfamily struct}}
\newcommand{\kwvoid}{\mbox{\color{myviolet}\ttfamily void}}
\newcommand{\kwint}{\mbox{\color{myviolet}\ttfamily int}}
\newcommand{\kwCreturn}{\mbox{\color{myviolet}\ttfamily return}}
\newcommand{\kwcase}{\mbox{\color{myviolet}\ttfamily case}}
\newcommand{\kwdefault}{\mbox{\color{myviolet}\ttfamily default}}
\newcommand{\kwenum}{\mbox{\color{myviolet}\ttfamily enum}}
\newcommand{\kwstatic}{\mbox{\color{myviolet}\ttfamily static}}

\newcommand{\figref}[1]{Figure~\ref{#1}}
\newcommand{\secref}[1]{Section~\ref{#1}}
\newcommand{\appref}[1]{Appendix~\ref{#1}}

\DeclareMathOperator{\passign}{passign}

\DeclareMathOperator{\fvarsop}{fvars}
\newcommand{\fvars}[1]{\fvarsop\BRA{#1}}
\newcommand{\fvarsd}[1]{\fvarsop'\BRA{#1}}

\DeclareMathOperator{\fargsop}{fargs}
\newcommand{\fargs}[1]{\fargsop\BRA{#1}}
\newcommand{\fargsd}[1]{\fargsop'\BRA{#1}}

\DeclareMathOperator{\Aop}{A}
\DeclareMathOperator{\Bop}{B}
\newcommand{\A}[2]{\Aop_{#1}\BRA{#2}}
\newcommand{\B}[2]{\Bop_{#1}\BRA{#2}}

\newcommand{\AbreakEq}[3]{\Aop_{#1}\llbracket \begin{aligned}[t] & #2 \\ & /\, #3 \rrbracket = \end{aligned}}
\newcommand{\BbreakEq}[3]{\Bop_{#1}\llbracket \begin{aligned}[t] & #2 \\ & /\, #3 \rrbracket = \end{aligned}}

\newcommand{\AbreakEqn}[2]{\Aop_{#1}\llbracket \begin{aligned}[t] & #2 \\ & \rrbracket = \end{aligned}}
\newcommand{\BbreakEqn}[2]{\Bop_{#1}\llbracket \begin{aligned}[t] & #2 \\ & \rrbracket = \end{aligned}}

\DeclareMathOperator{\Fop}{F}
\newcommand{\F}[1]{\Fop\BRA{#1}}
\DeclareMathOperator{\BRop}{BR}
\newcommand{\BR}[3]{\BRop\BRA{#1}_{#2,#3}}
\DeclareMathOperator{\Eop}{E}
\newcommand{\E}[3]{\Eop\BRA{#1\,/\,#2}_{#3}}

% based on the Coq reference manual, doc/common/macros.tex
\newcommand{\kw}[1]{\textsf{#1}}
\newcommand{\WF}[2]{\ensuremath{{\mathcal{W\!F}}(#1)[#2]}}
\newcommand{\WFTWOLINES}[2]{\ensuremath{{\mathcal{W\!F}}\begin{array}{l}(#1)\\\mbox{}[{#2}]\end{array}}}
\newcommand{\WFE}[1]{\WF{E}{#1}}
\newcommand{\WT}[4]{\ensuremath{#1[#2] \vdash #3 : #4}}
\newcommand{\WTE}[3]{\WT{E}{#1}{#2}{#3}}
\newcommand{\WTEG}[2]{\WTE{\Gamma}{#1}{#2}}
\newcommand{\subst}[3]{#1\{#2/#3\}}
\newcommand{\substm}[3]{#1\{\overline{#2/#3}\}}

% taken from the Coq reference manual, doc/sphinx/refman-preamble.sty
\newcommand{\case}{\kw{case}}
\newcommand{\Fix}{\kw{Fix}}

\newcommand{\vdashb}{\vdash_{\textrm{b}}}
\newcommand{\vdashf}{\vdash_{\textrm{f}}}

\newcommand{\reltri}{\mathrel{\triangleright}}

\newcommand{\rep}[1]{\overline{#1}}
\newcommand{\repop}[2]{\overline{#1\underbracket[0.4pt][1pt]{#2}}}
\newcommand{\repopi}[3]{\overline{#1\underbracket[0.4pt][1pt]{#2}}^{#3}}
\newcommand{\repi}[2]{\overline{#1}^{#2}}

\title{codegen development memo}

\begin{document}

\maketitle

\section{Notations}\label{sec:notations}

\subsection{Parenthesis}

We use parenthesis to represent grouping of syntax trees.
This is not restricted for expressions but for any syntax trees.
For example, we may use parenthesis as $\lam{(x{:}T)}{t}$ for $\lam{x{:}T}{t}$.
($x{:}T$ is not an expression.)

We don't include parenthesis in syntax definition in BNF.

\subsection{Repetition}

\begin{itemize}
  \item We use overline to represent repetition:
    $\repopi{x_i+y_i}{<}{1\leq i \leq n}$ means $x_1+y_1 < \ldots < x_n+y_n$.
  \item If the operator part ($\underbracket[0.4pt][1pt]{<}$) is just a punctuation to separate each term,
    we consider the punctuation can be added at first and/or last if appropriate for the context.
    $\repopi{x_i}{,}{1\leq i \leq n}$ represents
    ``$x_1,\ldots ,x_n$'',
    ``$x_1,\ldots ,x_n,$'',
    ``$,x_1,\ldots ,x_n$'', or
    ``$,x_1,\ldots ,x_n,$''. \\
    Thus, $(\repopi{x_i}{,}{1\leq i \leq m}\;\repopi{y_i}{,}{1\leq i \leq n})$ can represent \\
    $(x_1,\ldots, x_m, y_1,\ldots, y_n)$ when $(0<m) \wedge (0<n)$, \\
    $(x_1,\ldots, x_m)$ when $(0<m)\wedge(n=0)$, \\
    $(y_1,\ldots, y_n)$ when $(m=0)\wedge(0<n)$, and \\
    $()$ when $m=n=0$.
  \item We may omit the operator part when the operator is just a punctuation to separate each term.
    We write $\repi{x_i+y_i}{1\leq i \leq n}$ for $x_1+y_1, \ldots, x_n+y_n$ if comma is appropriate separator for the context.
    (Also, comma can be added at first and/or end as described above.)
  \item We may write the range part, $1\leq i \leq n$, as $i=1\ldots n$.
  \item We may write the range part as only an index metavariable if the range is clear from the context: $\repi{x_i+y_i}{i}$.
  \item We may omit the range part ($1\leq i \leq n$) and index of metavariables ($i$ of $x_i$ and $y_i$) when the metavariables are sequences of same length.
    We write $\repop{x+y}{<}$ for $x_1+y_1 < \ldots < x_n+y_n$ when $x$ and $y$ are $n$-element sequences.
  \item We omit both the operator and range part if appropriate.
    We write $\rep{x+y}$ for $x_1+y_1, \ldots, x_n+y_n$ if $x$ and $y$ are $n$-element sequences and comma is appropriate separator for the context.
    We use this form in most case.
  \item We use underline to distinguish metavariables which index is added by overline or not.
    We write $\rep{x+\underline{y}}$ for $x_1+y, \ldots, x_n+y$.
  \item We use nested overline to represent multi-dimensional indexes. \\
    $\omatch{t}{\rep{C\:\rep{x}\Rightarrow u}}$ means \\
    $\omatch{t}{C_1\:\rep{x_1}\Rightarrow u_1 \:|\: \ldots \:|\: C_n\:\rep{x_n}\Rightarrow u_n}$ and \\
    $\omatch{t}{C_1\: x_{11}\ldots x_{1m_1} \Rightarrow u_1
                \:|\: \ldots
                \:|\: C_n\: x_{n1}\ldots x_{nm_n} \Rightarrow u_n}$.
  \item Nested overline and underline can be combined. \\
    $\rep{\letinB{f'}{\ofix{\underline{(\rep{f:=t})}}{f}}}$ means \\
    $\letinB{f'_1}{\ofix{\rep{f:=t}}{f_1}} ~\ldots~
     \letinB{f'_h}{\ofix{\rep{f:=t}}{f_h}}$ and \\
    $\letinB{f'_1}{\ofix{(f_1:=t_1)\ldots(f_n:=t_n)}{f_1}} ~\ldots~
     \letinB{f'_h}{\ofix{(f_1:=t_1)\ldots(f_n:=t_n)}{f_h}}$.

    Overlines and underlines must construct a nested structure.
    If an underline and an overline covers same range,
    we consider the underline covers the overline.
    For example,
    we consider $\rep{a \rep{\underline{b}} c}$ as
    $\rep{a \underline{(\rep{b})} c} = a_1 \rep{b} c_1 \ldots a_n \rep{b} c_n$.
    We don't consider it as
    $\rep{a \rep{(\underline{b})} c} = a_1 \rep{(\underline{b_1})} c_1 \ldots a_n \rep{(\underline{b_n})} c_n$.
    We cannot define how to repeat $\rep{(\underline{b})}$ because it has no variable without underline.

  \item This notation is taken from \cite{steele2017s}.
\end{itemize}

\subsection{Number of Elements}

We use $|x|$ to represent the number of elements: $|x|=n$ if $x$ is an $n$-element sequence, $x_1, \ldots, x_n$.

\subsection{Number of Arguments}

\begin{itemize}
  \item $\NA_t$ is the number of arguments of $t$: \quad $\NA_t=m$ if $t : T_1 \rightarrow \dotsb \rightarrow T_m \rightarrow T_0$ and $T_0$ is an inductive type.
  \item $\NP_I$ is the number of the parameters of the inductive type $I$: \\
    $\NP_I=p$ if $I$ is defined in an inductive definitions $\text{\sf Ind}\:[p]\:(\Gamma_I := \Gamma_C)$.
  \item $\NI_I$ is the number of the indexes of the inductive type $I$ (the number of arguments without the parameters for the inductive type): \\
    $\NI_I=|\arr{I}|$ where
    $\arr{t}$ is the arity of the inductive type $t$.
    It means $(t : \forall (\Gamma_P; \arr{t}), S)$ is defined in $\Gamma_I$ of $\text{\sf Ind}\:[p]\:(\Gamma_I := \Gamma_C)$ in the global environment where
    $|\Gamma_P|$ = $p$ and $S$ is a sort.
  \item $\NM_C$ is the number of the members of the constructor $C$ (the number of arguments without the parameters for the inductive type): \\
    $\NM_C=|\Gamma|$ where
    $\Gamma$ is the non-parameter arguments of the constructor $C$.
    It means $(t : \forall (\Gamma_P; \Gamma), S)$ is defined in $\Gamma_C$ of $\text{\sf Ind}\:[p]\:(\Gamma_I := \Gamma_C)$ in the global environment where
    $|\Gamma_P|$ = $p$ and $S$ is a sort.
\end{itemize}

\subsection{Substitution}
$\subst{t}{x}{u}$ means a term in which variable $x$ in term $t$ is replaced by term $u$.
This notation is taken from the Coq reference manual~\cite{coqrefman8.12.0}.

We use $\substm{t}{x}{u}$ for parallel substituion.

\section{\gallina{}}\label{sec:gallina}
\subsection{\gallina{} Syntax}\label{sec:gallina-syntax}

\begin{align*}
  t &= x & \text{variable} \\
    &\bnfor c & \text{constant} \\
    &\bnfor C & \text{constructor} \\
    &\bnfor T & \text{type} \\
    &\bnfor \lamT{x}{T}{t}        & \text{abstraction} \\
    &\bnfor t\:u                  & \text{application} \\
    &\bnfor \letin{x}{t:T}{u}     & \text{let-in} \\
    &\bnfor \omatch{t}{\repop{C\:\rep{x{:}T}\Rightarrow u}{|}} & \text{conditional} \\
    &\bnfor \ofix{\repop{f/k{:}T:=t}{\kwwith}}{f_j} & \text{fixpoint}
\end{align*}
{\small Note:
\begin{itemize}
  \item $u, a, b$ represents a term as $t$. \\
    $v, w, y, z,$ and $f$ represent a variable as $x$. \\
    $U, V$ represents a type as $T$.
  \item We write $(\cdots((t\:u_1)\:u_2)\cdots\:u_n)$ as $t\:u_1\ldots u_n$ or $t\:\rep{u}$.
  \item We write $\lamT{x_1}{T_1}{(\ldots (\lamT{x_n}{T_n}{u}) \ldots)}$ as $\lamTM{x}{T}{u}$.
  \item We write $\letin{x_1}{t_1{:}T_1}{(\ldots(\letin{x_n}{t_n{:}T_n}{u})\ldots)}$ as $\letinM{x}{t{:}T}{u}$.
  \item $k$ is an integer. \\ $k_i$ for fixpoint specify the decreasing argument for $f_i$.
  \item If it is unambiguous, we omit type annotations for the sake of simplicity.  We also omit $k_i$ in fixpoints if they are not used.
  \item We omitted the elimination predicate (\kwas-\kwin-\kwreturn{} clause of \kwmatch-expression) in the syntax.
  \item We omitted the dummy parameters (underscores between $C$ and $\rep{x{:}T}$) in conditionals.
  \item We consider inductive types and constructor types has no let-in in binders.
  \item We omitted the detail of the types.  Actual \gallina{} permits any \gallina{} term which evaluates to a type.
\end{itemize}}

% {\tiny
% We ignore Var, Meta, Evar because they are not used in complete program.
% Int and Float are considered as constants.
% Prod, Ind and Sort are considered as types.
% Cast is ignored because it can be eliminated immediately.
% CoFix is ignored because lazy-evaluation is not suitable to C.
% Proj is ignored because it is similar to \kwmatch.}

\subsection{\gallina{} Conversion Rules}\label{sec:conversion-rules}

% based on sphinx/language/core/conversion.rst and doc/sphinx/language/core/inductive.rst
\begin{gather*}
  \text{beta:}~
    E[\Gamma] \vdash ((\lam{x}{t})\:u) \reltri \subst{t}{x}{u} \\
  \text{delta-local:}~
    \dfrac{(x:=t) \in \Gamma}{E[\Gamma] \vdash x \reltri t} \\
  \text{delta-global:}~
    \dfrac{(c:=t) \in E}{E[\Gamma] \vdash c \reltri t} \\
  \text{zeta:}~
    E[\Gamma] \vdash \letin{x}{t}{u} \reltri \subst{u}{x}{t} \\
  \text{iota-match:}~
    \dfrac
    {
      E[\Gamma] \vdash C_j\:\rep{a}\:\rep{b} : T \quad
      \text{$|a| = \NP_T$}
    }{
      E[\Gamma] \vdash
      \omatch{(C_j\:\rep{a}\:\rep{b})}{\rep{C\:\rep{x}\Rightarrow t}}
      \reltri
      (\lam{\rep{x_j}}{t_j})\:\rep{b}
    } \\
  \text{iota-fix:}~
    \dfrac{
      u_{k_j} = C\:\rep{a} \quad
      |u| = k_j
    }{
      E[\Gamma] \vdash\; (\ofix{\rep{f/k:=t}}{f_j})\:\rep{u} \reltri\; \substm{t_j}{f}{\ofix{\underline{(\rep{f/k:=t})}}{f}} \: \rep{u}
    } \\
  \text{eta expansion:}~
    \dfrac{\WTEG{t}{\prodT{x}{T}{U}}}{E[\Gamma] \vdash t \reltri \lamT{x}{T}{(t\:x)}}
\end{gather*}
{\small Note:
\begin{itemize}
  \item The rules shown here are reductions, except the eta expansion.
  \item Variables cannot conflict because \coq{} uses de Bruijn's indexes to represent variables.
  \item If it is unambiguous, we omit type annotations in these definitions for the sake of simplicity.
  \item Iota-match reduces $\kwmatch\:\mathtt{@cons\:nat\:1\:nil}\:\kwwith\:(\mathtt{nil} \Rightarrow t_1)\:|\:(\mathtt{cons}\:\mathtt{h}\:\mathtt{t} \Rightarrow t_2)\:\kwend$ to $(\lam{\mathtt{h}}{\lam{\mathtt{t}}{t_2}})\:1\:\mathtt{nil}$
    because \lstinline!list! has one parameter $(\NP_\mathtt{nat}=1)$ and \lstinline!cons! has two members $(\NM_\mathtt{cons}=2)$.
\end{itemize}}

\subsection{Free Variables}

$\FV(t)$ is the free variables of $t$.

\begin{align*}
  \FV(x) &= \{x\} \\
  \FV(c) &= \varnothing \\
  \FV(C) &= \varnothing \\
  \FV(T) &= \varnothing \\
  \FV(\lamT{x}{T}{t}) &= \FV(t) - \{x\} \\
  \FV(t\:u) &= \FV(t) \cup \FV(u) \\
  \FV(\letin{x}{t:T}{u}) &= \FV(t) \cup \FV(u) - \{x\} \\
  \FV(\omatch{t}{\rep{C\:\rep{x{:}T}\Rightarrow u}}) &= \FV(t) \cup \tbigcup_i (\FV(u_i) - \{\rep{x_i}\}) \\
  \FV(\ofix{\rep{f/k{:}T:=t}}{f_j}) &= \big(\tbigcup_i \FV(t_i)\big) - \{\rep{f}\}
\end{align*}
{\small Note:
\begin{itemize}
  \item We assume $\FV(T) = \varnothing$ because we consider non-dependent expressions.
\end{itemize}}

\subsection{Global Context and Local Context}

$E$ is a global environment which is a list of
global assumptions $\gassum{c}{T}$,
global definitions $\glodef{c}{t}{T}$, and
inductive definitions ($\text{\sf Ind}\:[p]\:(\Gamma_I := \Gamma_C)$).

$\Gamma$ is a local context which is a list of
local assumptions $\lassum{x}{T}$ and
local definitions $\ldef{x}{t}{T}$.
The local assumptions represent variables bounded by outer abstractions, conditionals, and fixpoints.
The local definitions represent variables bounded by outer let-in.

\subsection{Syntactic Context}

We use syntactic context $K$.
$K$ is a single-hole context: a \gallina{} term with a subterm is sustituted with a hole, $[]$.
$K[u]$ is $K$ with the hole is substituted with $u$.

We call syntactic context just as context if it is not ambiguous.

\begin{align*}
  K &= [] \\
    &\bnfor \lamT{x}{T}{K} \\
    &\bnfor K\:t \\
    &\bnfor t\:K \\
    &\bnfor \letin{x}{K:T}{t} \\
    &\bnfor \letin{x}{t:T}{K} \\
    &\bnfor \omatch{K}{\rep{C\:\rep{x{:}T}\Rightarrow u}} \\
    &\bnfor \omatch{t}{\repi{(C_l\:\rep{x_l{:}T_l}\Rightarrow u_l)}{1\leq l < i} (C_i\:\rep{x_i{:}T_i}\Rightarrow K) \repi{(C_l\:\rep{x_l{:}T_l}\Rightarrow u_l)}{i < l \leq h}  } \\
    &\bnfor \ofix{\repi{(f_l/k_l{:}T_l:=t_l)}{1\leq l < i} (f_i/k_i{:}T_i:=K) \repi{(f_l/k_l{:}T_l:=t_l)}{i < l \leq h}}{f_j}
\end{align*}

\subsection{Local Context of Syntactic Context}

$\LC(K)$ is the local context of the hole of $K$.
\begin{align*}
  \LC([]) &= \text{empty} \\
  \LC(\lamT{x}{T}{K}) &= \lassum{x}{T};  \LC(K) \\
  \LC(K\:t) &= \LC(K) \\
  \LC(t\:K) &= \LC(K) \\
  \LC(\letin{x}{K:T}{t}) &= \LC(K) \\
  \LC(\letin{x}{t:T}{K}) &= \ldef{x}{t}{T}; \LC(K) \\
  \LC(\omatch{K}{\rep{C\:\rep{x{:}T}\Rightarrow u}}) &= \LC(K) \\
  \LC(\omatch{t}{\repi{(C_l\:\rep{x_l{:}T_l}\Rightarrow u_l)}{1\leq l < i} (C_i\:\rep{x_i{:}T_i}\Rightarrow K) \repi{(C_l\:\rep{x_l{:}T_l}\Rightarrow u_l)}{i < l \leq h}} &= \rep{\lassum{x_i}{T_i};} \LC(K) \\
  \LC(\ofix{\repi{(f_l/k_l{:}T_l:=t_l)}{1\leq l < i} (f_i/k_i{:}T_i:=K) \repi{(f_l/k_l{:}T_l:=t_l)}{i < l \leq h}}{f_j}) &= \rep{\lassum{f_i}{T_i};} \LC(K)
\end{align*}

$\KV(K)$ is the bound variables usable in the hole of the context $K$.
\[
  \KV(K) = \{\, x \;|\; (\exists T. \lassum{x}{T} \in \LC(K)) \vee (\exists t,T. \ldef{x}{t}{T} \in LC(K)) \}
\]

\section{CodeGen}\label{sec:codegen}

\begin{itemize}
\item \gallina-to-\gallina{} Transformations
  \begin{itemize}
  \item Inlining
  \item Strip Cast
  \item Eta Expansion for Functions
  \item V-Normalization
  \item S-Normalization
  \item Type Normalization
  \item Static Argument Normalization
  \item Unused let-in Deletion
  \item Call Site Replacement
  \item Eta Reduction to Expose Fixpoint
  \item Argument Completion
  \item Monomorphism Check
  \item Borrow Check
  \item C Variable Allocation
  \end{itemize}
\item C Code Generation
  \begin{itemize}
  \item C Code Generation
  \end{itemize}
\end{itemize}

\section{\gallina-to-\gallina{} Transformations}\label{sec:gallina-to-gallina-transformations}

We define transformations as a judgement $E[\Gamma] \vdash t \reltri u$.
This means a subterm $t$ is substituted to $u$ where
$E$ and $\Gamma$ are the global environment and the local context of them.

We also use $E[\Gamma] \vdash_K t \reltri u$ to represent transformations restricted with a syntactical context $K$.

$E[\Gamma] \vdash_K t \reltri u$ is similar to $E[\Gamma] \vdash K[t] \reltri K[u]$ but
$\Gamma$ is the local context of $t$ (not $K[t]$).

Also, we use $E[] \vdash_\$ t \reltri u$ which defines a tranformation of an entire term (not subterm).
(The local context is empty because an entire term has no local context.)

When we define a new constant in a transformation,
We use $E[\Gamma] \vdash t \reltri (E;\glodef{c}{a}{T})[\Gamma] \vdash u$.

\subsection{Inlining}\label{sec:inlining}

\codegen{} apply delta-global reductions to inline definitions.

Two command, \lstinline!CodeGen Global Inline! and \lstinline!CodeGen Local Inline!, specifies
what definitions will be inlined.
\begin{lstlisting}
CodeGen Global Inline QUALID...
CodeGen Local Inline QUALID : QUALID...
\end{lstlisting}

\lstinline[mathescape=true]!CodeGen Global Inline $c_1\ldots c_n$! specifies
global constants $c_1\ldots c_n$ will be expanded.

\lstinline[mathescape=true]!CodeGen Local Inline $c_0$ : $c_1\ldots c_n$! specifies
global constants $c_1\ldots c_n$ will be expanded in $c_0$.

\subsection{Strip Cast}\label{sec:strip-cast}

\subsection{Eta Expansion for Functions}\label{sec:eta-expand-funcs}
We apply eta-expansion to functions of top-level functions, fix-bounded functions, and closure generating lambdas.
We consider explicit lambdas are closure generation.
This makes beta-var applicable for partial applications without worrying to expose computation.
(CIC~\cite{coqrefman8.12.0} uses lambdas for match-branches but our syntax uses no lambdas for them.
Thus match-branches doesn't trigger the eta expansion.)

\begin{gather*}
  \text{etaex-top:}~
    \dfrac{
      E[] \vdash t : \prodT{x}{T}{U} \quad
      \text{$t$ is not an abstraction nor fixpoint} \quad
    }{
      E[] \vdash_\$ t \reltri \lamT{x}{T}{(t\:x)}
    } \breakrule
  \text{etaex-fix:}~
    \dfrac{
      \begin{aligned}
        & E[\Gamma] \vdash t_i : \prodT{x}{T}{U} \quad
          \text{$t_i$ is not an abstraction nor fixpoint} \\
        & K = \ofix{\repi{(f_l:=t_l)}{1\leq l < i} (f_i:=[]) \repi{(f_l:=t_l)}{i < l \leq h}}{f_j}
      \end{aligned}
    }{
        E[\Gamma] \vdash_K t_i \reltri \lamT{x}{T}{(t_i\:x)}
    } \breakrule
  \text{etaex-abs:}~
    \dfrac{
      \begin{aligned}
        & E[\Gamma] \vdash t : \prodT{x}{T}{U} \quad
          \text{$t$ is not an abstraction nor fixpoint} \\
        & K = \lam{y}{[]}
      \end{aligned}
    }{
      E[\Gamma] \vdash_K t \reltri \lamT{x}{T}{(t\:x)}
    }
\end{gather*}

This transformations makes a term in following syntax.
The entire term is represented as $\tD'$.
($\tD$ for functions and $\tE$ for non-function constants.)

\begin{align*}
  \tD &= \lamT{x}{T}{\tD'} \\
      &\bnfor \ofix{\rep{f/k{:}T:=\tD}}{f_j} \\
  \tD' &= \tD \\
       &\bnfor \tE & \text{the type of $\tE$ is an inductive type} \\
  \tE &= x \\
    &\bnfor c \\
    &\bnfor C \\
    &\bnfor T \\
    &\bnfor \tE\:\tE \\
    &\bnfor \letin{x}{\tE:T}{\tE} \\
    &\bnfor \omatch{\tE}{\rep{C\:\rep{x}\Rightarrow \tE}} \\
    &\bnfor \tD \\
\end{align*}

The type of $\tE$ in $\tD'$ is an inductive type because
the eta expansions (etaex-fix and etaex-abs) transform the body of abstraction and fixpoint until its type is not function type.

\subsection{V-Normalization}\label{sec:v-normalization}
\subsubsection{V-Reductions}\label{sec:v-reductions}
\begin{gather*}
  \text{zeta-arg:}~
    \dfrac
    {
      E[\Gamma] \vdash u : U \quad
      \text{$t$ is not an application} \quad
      \text{$u$ is not a variable} \quad
      \text{$y$ is a fresh variable}
    }{
      E[\Gamma] \vdash
      t\:\rep{x}\:u\:\rep{a}
      \reltri
      \letin{y}{u:U}{t\:\rep{x}\:y\:\rep{a}}
    } \breakrule
  \text{zeta-item:}~
    \dfrac
    {
      E[\Gamma] \vdash u : U \quad
      \text{$u$ is not a variable} \quad
      \text{$y$ is a fresh variable}
    }{
      E[\Gamma] \vdash
        \omatch{u}{\rep{C\:\rep{x} \Rightarrow t}}
        \reltri
        \letin{y}{u:U}{\omatch{y}{\rep{C\:\rep{x} \Rightarrow t}}}
    }
\end{gather*}

\subsubsection{V-Normal Form}\label{sec:v-normal-form}
V-normal form restricts \gallina{} terms that (1) application arguments and (2) match items to variables.
\begin{align*}
  t &= x \bnfor c \bnfor C \bnfor T \bnfor \lamT{x}{T}{t} \bnfor \letin{x}{t:T}{u} \\
    &\bnfor \ofix{\rep{f/k{:}T:=t}}{f_j} \\
    &\bnfor t\:x                               & \leftarrow (1) \\
    &\bnfor \omatch{x}{\rep{C\:\rep{x} \Rightarrow t}} & \leftarrow (2)
\end{align*}

Since we apply V-reductions for a eta-expanded term,
the result term can be represented in following syntax.

\begin{align*}
  \tD &= \lamT{x}{T}{\tD'} \\
      &\bnfor \ofix{\rep{f/k{:}T:=\tD}}{f_j} \\
  \tD' &= \tD \\
       &\bnfor \tE & \text{the type of $\tE$ is an inductive type} \\
  \tE &= x \\
    &\bnfor c \\
    &\bnfor C \\
    &\bnfor T \\
    &\bnfor \tE\:x & \leftarrow (1) \\
    &\bnfor \letin{x}{\tE:T}{\tE} \\
    &\bnfor \omatch{x}{\rep{C\:\rep{x}\Rightarrow \tE}} & \leftarrow (2) \\
    &\bnfor \tD \\
\end{align*}

\subsection{S-Normalization}\label{sec:s-normalization}

\subsubsection{S-Reductions}\label{sec:s-reductions}
\begin{gather*}
  \text{beta-var:}~
    E[\Gamma] \vdash (\lam{x}{t})\:y \reltri \subst{t}{x}{y} \breakrule
  \text{delta-var:}~
    \dfrac{(x:=y) \in \Gamma}{E[\Gamma] \vdash x \reltri y} \breakrule
  \text{delta-fun:}~
    \dfrac
    {
      \begin{gathered}
      0 \leq |x| \quad
      0 < |y| \quad
      (f := t\:\rep{x}) \in \Gamma \\
      \text{$t$ is one of variable, constant, constructor, abstraction, or fixpoint}
      \end{gathered}
    }{
      E[\Gamma] \vdash f\:\rep{y}
                       \reltri
                       t\:\rep{x}\:\rep{y}
    } \breakrule
  \text{zeta-flat:}~
    E[\Gamma] \vdash \letin{y}{(\letin{x}{t_1}{t_2})}{t_0}
                       \reltri
                       \letin{x}{t_1}{(\letin{y}{t_2}{t_0})} \breakrule
  \text{zeta-app:}~
    E[\Gamma] \vdash
     (\letin{y}{t}{u})\:\rep{x}
     \reltri
     \letin{y}{t}{(u\:\rep{x})} \breakrule
  \text{iota-match-var:}~
    \dfrac
    {
      (v:=C_j\:\rep{y}\:\rep{z}:T) \in \Gamma \quad
      |y|=\NP_T
    }{
      E[\Gamma] \vdash
      \omatch{v}{\rep{C\:\rep{x}\Rightarrow t}}
      \reltri
      (\lam{\rep{x_j}}{t_j})\:\rep{z}
    } \breakrule
  \text{iota-fix-var:}~
    \dfrac
    {
      \begin{gathered}
        (x_{k_j} := C\:\rep{y}) \in \Gamma \quad
        \text{$\rep{f'}$ are fresh variables} \\
        E[\Gamma] \vdash (\ofix{\rep{f/k:=t}}{f_j})\:\rep{x} : T \quad
        \text{$T$ is an inductive type}
      \end{gathered}
    }{
        E[\Gamma] \vdash\;
          (\ofix{\rep{f/k:=t}}{f_j})\:\rep{x}
          \reltri
          \letinM{f'}{\ofix{\underline{(\rep{f/k:=t})}}{f}}{\substm{t_j}{f}{f'}} \: \rep{x}
    } \breakrule
  \text{iota-fix-var':}~
    \dfrac
    {
      \begin{gathered}
        (x_{k_j} := C\:\rep{y}) \in \Gamma \quad
        \rep{(f' := \ofix{\underline{(\rep{f/k:=t})}}{f}) \in \underline{\Gamma}} \\
        E[\Gamma] \vdash (\ofix{\rep{f/k:=t}}{f_j})\:\rep{x} : T \quad
        \text{$T$ is an inductive type}
      \end{gathered}
    }{
      E[\Gamma] \vdash
      (\ofix{\rep{f/k:=t}}{f_j})\:\rep{x}
      \reltri
      \substm{t_j}{f}{f'} \:\rep{x}
    } \breakrule
  \text{match-app:}~
    \dfrac
    {
      E[\Gamma] \vdash z : T
    }{
      \begin{aligned}
        E[\Gamma] \vdash\; &
          \kwmatch\:x\:
          \kwas\:x'\:
          \kwin\:I\:\rep{y}\:
          \kwreturn\:{T \rightarrow P\:\rep{y}\:x'}
          \kwwith\:\rep{{C\: \rep{x}} \Rightarrow {t}}\:
          \kwend\:z \\
        \reltri\; &
          \kwmatch\:x\:
          \kwas\:x'\:
          \kwin\:I\:\rep{y}\:
          \kwreturn\:{P\:\rep{y}\:x'}
          \kwwith\:\rep{{C\: \rep{x}} \Rightarrow {t\:\underline{z}}}\:
          \kwend
      \end{aligned}
    }
\end{gather*}

{\small Note:
\begin{itemize}
  \item match-app is not convertible
\end{itemize}}

\subsubsection{S-Normal Form}\label{sec:s-normal-form}
S-reductions transform applications to restrict function positions.
\begin{itemize}
  \item beta-var removes an abstraction at the function position of an application.
  \item zeta-app removes a let-in at the function position of an application.
  \item match-app removes a conditional at the function position of an application.
\end{itemize}
Also, types cannot be a function.
We treat multi-arguments application as single application, application is not occur at a function position.
Thus, function position can be variable, constant, construcor, or fixpoint in the S-normal form.

Also, zeta-flat removes a let-in at the binder term of a let-in.

\begin{align*}
  \tD &= \lamT{x}{T}{\tD'} \\
      &\bnfor \ofix{\rep{f/k{:}T:=\tD}}{f_j} \\
  \tD' &= \tD \\
       &\bnfor \tL & \text{the type of $\tL$ is an inductive type} \\
  \tL &= \letinM{x}{\tM:T}{\tM} \\
  \tM &= \omatch{x}{\rep{C\:\rep{x}\Rightarrow \tL}} \\
      &\bnfor \tE \\
  \tE &= x \bnfor c \bnfor C \bnfor T \\
    &\bnfor \tF\:\rep{x} & \text{$0 < |x|$} \\
    &\bnfor \tD \\
  \tF &= x \bnfor c \bnfor C \bnfor \ofix{\rep{f/k{:}T:=\tD}}{f_j}
\end{align*}

\subsection{Type Normalization}\label{sec:type-normalization}

We normalize type annotations in the term.

This transformation makes that types contain no variable bounded by let-ins
because such variables are redex of delta reduction.
Thus, this transformation makes Unused let-in Deletion (\secref{sec:let-in-deletion}) more effective.

\begin{align*}
  \tD &= \lamT{x}{\fbox{$T$}}{\tD'} & \leftarrow \\
      &\bnfor \ofix{\rep{f/k{:}\fbox{$T$}:=\tD}}{f_j} & \leftarrow \\
  \tD' &= \tD \\
       &\bnfor \tL & \text{the type of $\tL$ is an inductive type} \\
  \tL &= \letinM{x}{\tM:\fbox{$T$}}{\tM} & \leftarrow \\
  \tM &= \omatch{x}{\rep{C\:\rep{x}\Rightarrow \tL}} \\
      &\bnfor \tE \\
  \tE &= x \bnfor c \bnfor C \bnfor T \\
    &\bnfor \tF\:\rep{x} & \text{$0 < |x|$} \\
    &\bnfor \tD \\
  \tF &= x \bnfor c \bnfor C \bnfor \ofix{\rep{f/k{:}\fbox{$T$}:=\tD}}{f_j} & \leftarrow
\end{align*}

We also normalize types in \kwmatch-expressions to make less free variables.
\gallina{} internal representation of \kwmatch-expressions contains
parameters for the inductive type, return clause, and SProp inversion data.

\subsection{Static Argument Normalization}\label{sec:static-argument-normalization}

We normalize static arguments.
We assume the normalized static arguments have no free variables.

It makes the syntax as follows.

\begin{align*}
  \tD &= \lamT{x}{T}{\tD'} \\
      &\bnfor \ofix{\rep{f/k{:}T:=\tD}}{f_j} \\
  \tD' &= \tD \\
       &\bnfor \tL & \text{the type of $\tL$ is an inductive type} \\
  \tL &= \letinM{x}{\tM:T}{\tM} \\
  \tM &= \omatch{x}{\rep{C\:\rep{x}\Rightarrow \tL}} \\
      &\bnfor \tE \\
  \tE &= x \bnfor c \bnfor C \bnfor T \\
    &\bnfor \tF\:\rep{x} & \text{$0 < |x|$} \\
    &\bnfor \tC\:\rep{\tA} & \text{$0 < |\tA|$} \\
    &\bnfor \tD \\
  \tF &= x \bnfor \ofix{\rep{f/k{:}T:=\tD}}{f_j} \\
  \tC &= c \bnfor C \\
  \tA &= x \bnfor u & \text{$u$ is a static argument (normal \gallina{} term without FV)}
\end{align*}

The application in previous section, $\tF\:\rep{x}$, is changed to $\tC\:\rep{\tA}$ for constant and constructor applications.
(This is not V-normal form because $\tA$ can be non-variable.)

Static arguments are defined as follows by default.
\begin{itemize}
  \item non-monomorphic arguments for constant functions.
    (The non-monomorphic argument means an argument which type is a sort or a polymorphic function type.)
  \item parameters for constructors.
\end{itemize}
The static arguments can be configured with \lstinline!CodeGen Arguments! command.

This transformation makes that static arguments contain no variable bounded by let-ins
because such variables are redex of delta reduction.
Thus, this transformation makes Unused let-in Deletion (\secref{sec:let-in-deletion}) more effective.

\subsection{Unused let-in Deletion}\label{sec:let-in-deletion}

\begin{gather*}
  \text{zeta-del:}~
    \dfrac{
      \text{$x$ does not occur in $u$} \quad \text{$x$ is not linear} \quad \text{$\FV(t)$ does not contain linear variable}
    }{E[\Gamma] \vdash \letin{x}{t}{u}
                       \reltri
                       u
    }
\end{gather*}

\subsection{Call Site Replacement}\label{sec:call-site-replacement}

\begin{gather*}
  \text{replace:}~
    \dfrac{
      \begin{gathered}
        \text{$t$ is a constant or constructor} \\
        \text{$a_{|a|}$ is not a variable} \\
        \rep{a} = \merg_t(\rep{x}, \rep{u}) \\
        \text{$\rep{y}$ are fresh variables} \quad
        |x| = |y| \\
        \rep{b} = \merg_t(\rep{y}, \rep{u}) \\
        \text{$c$ is a fresh constant} \\
      \end{gathered}
    }{E[\Gamma] \vdash_K t\:\rep{a}\:\rep{z}
      \reltri
      (E;\glodefB{c}{\lam{\rep{y}}{t\:\rep{b}}})[\Gamma] \vdash c\:\rep{x}\:\rep{z}
    }
\end{gather*}

$K$ is a non-application context to restrict $t\:\rep{a}\:\rep{z}$ is not at a function position of an application.
($K = \$, (\lam{x}{[]})$, $(t\:[])$, $(\letin{x}{[]}{u})$, $(\letin{x}{t}{[]})$, or \ldots but NOT $([]\:u)$.)

$\merg_t(\rep{x}, \rep{u})$ represents a sequence of terms which two sequences of terms are merged according to the static arguments definition of $t$.
The first argument $\rep{x}$ specifies dynamic arguments.
The second argument $\rep{u}$ specifies static arguments.
For example, assuming the 1st and 4th arguments are static for $t$, $\merg_t((x_1, x_2, x_3), (u_1, u_2)) = (u_1, x_1, x_2, u_2, x_3)$.

This transformation removes non-variable arguments from applications.
Thus the result will be V-normal form again.

\begin{align*}
  \tD &= \lamT{x}{T}{\tD'} \\
      &\bnfor \ofix{\rep{f/k{:}T:=\tD}}{f_j} \\
  \tD' &= \tD \\
       &\bnfor \tL & \text{the type of $\tL$ is an inductive type} \\
  \tL &= \letinM{x}{\tM:T}{\tM} \\
  \tM &= \omatch{x}{\rep{C\:\rep{x}\Rightarrow \tL}} \\
      &\bnfor \tE \\
  \tE &= x \bnfor c \bnfor C \bnfor T \\
    &\bnfor \tF\:\rep{x} & \text{$0 < |x|$} \\
    &\bnfor \tD \\
  \tF &= x \bnfor c \bnfor C \bnfor \ofix{\rep{f/k{:}T:=\tD}}{f_j}
\end{align*}

\subsection{Eta Reduction to Expose Fixpoint}\label{sec:eta-reduction}

\begin{gather*}
  \text{etared-fix:}~
    \dfrac{
      E[\Gamma] \vdash t : \forall \rep{x{:}T}.\: U \quad
      \text{$\rep{x}$ does not occur in $t$} \quad
      \text{$t$ is a fixpoint}
    }{E[\Gamma] \vdash \lam{\rep{x{:}T}}{t\: \rep{x}}
                       \reltri
                       t
    }
\end{gather*}
{\small Note:
\begin{itemize}
  \item We require the types of arguments of $t$ as $\rep{T}$ to prevent this transfomation changes the type.
  \item The premise ``$t$ is a fixpoint'' guarantees the result is not partial application.
\end{itemize}}

This transformation is intended to remove eta-redexes introduced by static arguments and eta expansion (\secref{sec:eta-expand-funcs}).
For example, assume the standard list concatenation function,
$\textrm{app}: \forall A, \textrm{list}\:A \rightarrow \textrm{list}\:A \rightarrow \textrm{list}\:A$,
is monomorphized to $\textrm{bool}$.

\newcommand{\ttA}{\texttt{A}}
\newcommand{\ttT}{\texttt{T}}
\newcommand{\ttl}{\texttt{l}}
\newcommand{\ttm}{\texttt{m}}
\begin{align*}
  & \texttt{app}\:\texttt{bool} \\
  \reltri_\textrm{inline}\; & (\lam{\ttA}{(\kwfix \ldots)})\:\texttt{bool} \\
  \reltri_\textrm{eta-expansion}\; & \lam{\ttl}{\lam{\ttm}{(\lam{\ttA}{(\kwfix \ldots)})\:\texttt{bool}\:\ttl\:\ttm}} \\
  \reltri_\textrm{V-normalization}\; & \lam{\ttl}{\lam{\ttm}{\letin{\ttT}{\texttt{bool}}{(\lam{\ttA}{(\kwfix \ldots)})\:\ttT\:\ttl\:\ttm}}} \\
  \reltri_\textrm{S-normalization}\; & \lam{\ttl}{\lam{\ttm}{\letin{\ttT}{\texttt{bool}}{(\kwfix \ldots)\:\ttl\:\ttm}}} \\
  \reltri_\textrm{Type-nomalization}\; & \lam{\ttl}{\lam{\ttm}{\letin{\ttT}{\texttt{bool}}{(\kwfix \ldots)\:\ttl\:\ttm}}} & \textrm{(expand $\ttT$ in the \kwfix-term)} \\
  \reltri_\textrm{unused-letin-deletion}\; & \lam{\ttl}{\lam{\ttm}{(\kwfix \ldots)\:\ttl\:\ttm}} \\
  \reltri_\textrm{eta-reduction}\; & (\kwfix \ldots) \\
\end{align*}

The code generator (\secref{sec:c-code-gen}) generates multiple C functions (\secref{sec:genfunm}) from the pre-eta-reduction term, $\lam{\ttl}{\lam{\ttm}{(\kwfix \ldots)\:\ttl\:\ttm}}$.
This eta-reduction avoid this.
The code generator generates single C function (\secref{sec:genfuns}) from post-eta-reduction term, $(\kwfix \ldots)$.

\subsection{Argument Completion}\label{sec:argcomp}
Argument completion removes partial applications by applying eta expansions.

\begin{gather*}
  \text{argcomp-papp:}~
    \dfrac{
      \begin{gathered}
        \text{$t$ is not an application} \\
        0 < |x| \\
        E[\Gamma] \vdash t\:\rep{x} : \forall \rep{y{:}T}, U \\
        \text{$U$ is an inductive type}
      \end{gathered}
    }{E[\Gamma] \vdash_K
        t\:\rep{x}
        \reltri
        \lam{\rep{y{:}T}}{t\:\rep{x}\:\rep{y}}
    } \breakrule
  \text{argcomp-cnst-cstr:}~
    \dfrac{
      \begin{gathered}
        \text{$t$ is a constant or constructor} \\
        E[\Gamma] \vdash t : \forall \rep{y{:}T}, U \\
        \text{$U$ is an inductive type}
      \end{gathered}
    }{E[\Gamma] \vdash_K
        t
        \reltri
        \lam{\rep{y{:}T}}{t\:\rep{y}}
    }
\end{gather*}

$K$ is a non-application context as in \secref{sec:call-site-replacement}.

This transformation makes the result of an application inductive type.
Also, constants and constructors are always fully applied to arguments.

\begin{align*}
  \tD &= \lamT{x}{T}{\tD'} \\
      &\bnfor \ofix{\rep{f/k{:}T:=\tD}}{f_j} \\
  \tD' &= \tD \\
       &\bnfor \tL & \text{the type of $\tL$ is an inductive type} \\
  \tL &= \letinM{x}{\tM:T}{\tM} \\
  \tM &= \omatch{x}{\rep{C\:\rep{x}\Rightarrow \tL}} \\
      &\bnfor \tE \\
  \tE &= x \bnfor T \\
    &\bnfor c \bnfor C & \text{The type of $c$ and $C$ are inductive type} \\
    &\bnfor \tF\:\rep{x} & \text{$0 < |x|$, the type of $\tF\:\rep{x}$ is inductive type}  \\
    &\bnfor \tD \\
  \tF &= x \bnfor c \bnfor C \bnfor \ofix{\rep{f/k{:}T:=\tD}}{f_j}
\end{align*}

\subsection{Monomorphism Check}\label{sec:check-monomorphism}

We check the transformed term is a monomorphic term.

Although our transformations removes many rank-1 polymorphism,
it still possible to retain polymorphic term.
For example, our transformations don't remove polymorphic recursion unless
the recursion is completely unrolled.

This step checks
(1) all type annotations are inductive or function types without free variables, and
(2) types and sorts doesn't occur at expression.

\begin{align*}
  \tD &= \lamT{x}{T}{\tD'} \\
      &\bnfor \ofix{\rep{f/k{:}T:=\tD}}{f_j} \\
  \tD' &= \tD \\
       &\bnfor \tL & \text{the type of $\tL$ is an inductive type} \\
  \tL &= \letinM{x}{\tM:T}{\tM} \\
  \tM &= \omatch{x}{\rep{C\:\rep{x}\Rightarrow \tL}} \\
      &\bnfor \tE \\
  \tE &= x & \leftarrow \text{$T$ is removed} \\
    &\bnfor c \bnfor C & \text{The type of $c$ and $C$ are inductive type} \\
    &\bnfor \tF\:\rep{x} & \text{$0 < |x|$, the type of $\tF\:\rep{x}$ is inductive type}  \\
    &\bnfor \tD \\
  \tF &= x \bnfor c \bnfor C \bnfor \ofix{\rep{f/k{:}T:=\tD}}{f_j}
\end{align*}

\subsection{Borrow Check}\label{sec:borrow-check}

We define two judgements $E[\Gamma] \vdash t:T~|~B$ and $E[\Gamma] \vdash t:T~|~(L, B^\text{used}, B^\text{result})$ for borrow check.
We extend $\Gamma$ in this section.
$\Gamma$ is an annotated local context.
It is a list of $\lassum{x^B}{T}$ and $\ldef{x^B}{t}{T}$.
They are same as local assumption and local definitions except that
the variable $x$ is annotated with a borrow information $B$.
$B$ is a set of pair of borrow type and linear variable, such as $\{\rep{(T,x)}\}$.
$B^\text{used}$ and $B^\text{result}$ are also borrow information.
$L$ is a set of linear variables.
$T$ is the type of $t$.

We write $\Bop_\Gamma x$ to refer the borrow information for $x$ in $\Gamma$.
$\Bop_\Gamma x = B$ if $\Gamma$ contains $\lassum{x^B}{T}$ or $\ldef{x^B}{t}{T}$.

We omit $:T$ in a rule which does not use $T$.

The borrow information $B=\{\rep{(T,x)}\}$ represents a linear variable $x_i$ is used via borrow type $T_i$.
$\lassum{x^{\{(T',y)\}}}{T}\in \Gamma$ represents $x$ may contain a value of type $T'$ which is a (part of) content of the linear variable $y$.

$E[\Gamma] \vdash t~|~B$ means a function $t$ may use linear variables via borrow $B$.

$E[\Gamma] \vdash t~|~(L, B^\text{used}, B^\text{result})$ means an expression $t$
(1) consumes linear variables $L$,
(2) may use linear values via borrow $B^\text{used}$,
(3) result value may contain linear values via borrow $B^\text{result}$.

For example, assume linear list \lstinline!lseq!, borrow list \lstinline!bseq! which has constructors \lstinline!bnil! and \lstinline!bcons!,
borrow function \lstinline!borrow : lseq nat -> bseq nat!.
In a code fragment \\
\lstinline!let y := borrow x in match y with bnil => true | bcons h t => false end! contains variables
\lstinline!x : lseq nat!,
\lstinline!y : bseq nat!,
\lstinline!h : nat!, and
\lstinline!t : bseq nat!.
\texttt{y} and \texttt{t} contain a \texttt{bseq nat} value borrowed from \texttt{x}.
It is represented as
$\texttt{y}^{\{(\texttt{bseq nat},\texttt{x})\}}:\text{\texttt{bseq nat}}$ and
$\texttt{t}^{\{(\texttt{bseq nat},\texttt{x})\}}:\text{\texttt{bseq nat}}$.
\texttt{h} is annotated as $\texttt{h}^\varnothing$ which means \texttt{h} does not contain borrowed values.
The type of \texttt{h} is \texttt{nat}.
Since \texttt{nat} is not a borrow type, \texttt{h} lives even after \texttt{x} is consumed.

\begin{gather*}
  \text{borrow-lvar:}~
    \dfrac
    {
      \lassum{x^B}{T} \in \Gamma \quad \text{$x$ is linear}
    }{
      E[\Gamma] \vdash x ~|~ (\{x\}, B, B)
    } \breakrule
  \text{borrow-var:}~
    \dfrac
    {
      \lassum{x^B}{T} \in \Gamma \quad \text{$x$ is not linear}
    }{
      E[\Gamma] \vdash x ~|~ (\varnothing, B, B)
    } \breakrule
  \text{borrow-constant:}~
    \dfrac
    {
      \text{$c$ is not a borrow function}
    }{
      E[\Gamma] \vdash c ~|~ (\varnothing, \varnothing, \varnothing)
    } \breakrule
  \text{borrow-constructor:}~
    E[\Gamma] \vdash C ~|~ (\varnothing, \varnothing, \varnothing) \breakrule
  \text{borrow-letin:}~
    \dfrac
    {
      \begin{gathered}
        E[\Gamma] \vdash t_1 ~|~ (L_1, B^\text{used}_1, B^\text{result}_1) \\
        E[\Gamma;\ldef{x^{B^\text{result}_1}}{t_1}{T}] \vdash t_2 ~|~ (L_2, B^\text{used}_2, B^\text{result}_2) \\
        L_1 \cap L_2 = \varnothing \\
        \text{$x$ is linear} \rightarrow x \in L_2 \\
        L_1 \cap B^\text{used}_2 = \varnothing
      \end{gathered}
    }{
      E[\Gamma] \vdash \letin{x}{t_1:T}{t_2} ~|~ (L_1\cup L_2 - \{x\}, B^\text{used}_1\cup B^\text{used}_2 - \{x\}, B^\text{result}_2 - \{x\})
    } \breakrule
  \text{borrow-match:}~
    \dfrac{
      \begin{gathered}
        E[\Gamma] \vdash y ~|~(L_\text{item}, B^\text{used}_\text{item}, B^\text{result}_\text{item}) \\
        B_{ij} = B^\text{result}_\text{item} \cap \components_E(T_{ij}) \\
        \rep{\Gamma' = \rep{\lassum{x^{B}}{T}}} \\
        \rep{\underline{E}[\underline{\Gamma} ; \Gamma'] \vdash t ~|~(L, B^\text{used}, B^\text{result})} \\
        % M stands for member (of constructor).
        % F stands for free variable.
        \rep{L \cap \{\rep{x}\} = \{ z ~|~ z \in \{\rep{x}\} ~\wedge~ \text{$z$ is linear} \}} \\
        \rep{L^\text{F} = L - \{\rep{x}\}} \quad
        \tbigwedge_i L^\text{F}_1 = L^\text{F}_i \quad
        L_\text{item} \cap L^{F}_1 = \varnothing \\
        B^\text{used}_\text{branches} = \tbigcup_i ({B^\text{used}_i - \{\rep{x_i}\}}) \quad
        B^\text{result}_\text{branches} = \tbigcup_i (B^\text{result}_i - \{\rep{x_i}\}) \quad
        L_\text{item} \cap B^\text{used}_\text{branches} = \varnothing \\
      \end{gathered}
    }{
      E[\Gamma] \vdash \omatch{y}{\rep{C\: \rep{x{:}T} \Rightarrow t}} ~|~ (L_\text{item} \cup L^\text{F}_1, B^\text{used}_\text{item} \cup B^\text{used}_\text{branches}, B^\text{result}_\text{branches})
    } \breakrule
  \text{borrow-var-app:}~
    \dfrac{
      \begin{gathered}
        \lassum{f^{B}}{T'} \in \Gamma \quad
        \APP(E, \Gamma, B, \rep{x}, T, L, B^\text{used}, B^\text{result})
      \end{gathered}
    }{
      E[\Gamma] \vdash f\:\rep{x} : T ~|~ (L, B^\text{used}, B^\text{result})
    } \breakrule
  \text{borrow-constant-app:}~
    \dfrac{
      \begin{gathered}
        \text{$c$ is not a borrow function} \quad
        \APP(E, \Gamma, \varnothing, \rep{x}, T, L, B^\text{used}, B^\text{result})
      \end{gathered}
    }{
      E[\Gamma] \vdash c\:\rep{x} : T ~|~ (L, B^\text{used}, B^\text{result})
    } \breakrule
  \text{borrow-constructor-app:}~
    \dfrac{
      \APP(E, \Gamma, \varnothing, \rep{x}, T, L, B^\text{used}, B^\text{result})
    }{
      E[\Gamma] \vdash C\: \rep{x} : T ~|~ (L, B^\text{used}, B^\text{result})
    } \breakrule
  \text{borrow-fix-app:}~
    \dfrac{
      \begin{gathered}
        E[\Gamma] \vdash \ofix{\rep{f:=t}}{f_j} ~|~ B \quad
        \APP(E, \Gamma, B, \rep{x}, T, L, B^\text{used}, B^\text{result})
      \end{gathered}
    }{
      E[\Gamma] \vdash (\ofix{\rep{f:=t}}{f_j})\:\rep{x} : T ~|~ (L, B^\text{used}, B^\text{result})
    } \breakrule
  \text{borrow-borrow:}~
    \dfrac{
      \begin{gathered}
        \text{$c$ is a borrow function} \\
        E[\Gamma] \vdash c : T^\text{arg} \rightarrow T^\text{result} \quad
        \text{$T^\text{arg}$ is a linear type} \quad
        \text{$T^\text{result}$ is a borrow type} \\
        \text{$T^\text{result}$ does not contain function} \\
        \text{$\{\rep{T}\}$ is the set of borrow types contained in $T^\text{result}$} \\
        B = \{\rep{(T,\underline{x})}\}
      \end{gathered}
    }{
      E[\Gamma] \vdash c\:x ~|~ (\varnothing, B, B)
    } \breakrule
  \text{borrow-fix-clo:}~
    \dfrac{
      E[\Gamma] \vdash \ofix{\rep{f:=t}}{f_j} ~|~ B
    }{
      E[\Gamma] \vdash \ofix{\rep{f:=t}}{f_j} ~|~ (\varnothing, B, B)
    } \breakrule
  \text{borrow-abs-clo:}~
    \dfrac{
      E[\Gamma] \vdash \lam{x}{t} ~|~ B
    }{
      E[\Gamma] \vdash \lam{x}{t} ~|~ (\varnothing, B, B)
    }
\end{gather*}

\begin{gather*}
  \text{borrow-abs-fun:}~
    \dfrac{
      \begin{gathered}
        \text{$t$ is not an abstraction} \\
        \text{$t$ is not a fixpoint} \\
        E[\Gamma;\rep{\lassum{x^\varnothing}{T}}] \vdash t ~|~ (L,B^\text{used},B^\text{result}) \\
        \{ z | z \in \{\rep{x}\} \wedge \text{$z$ is linear} \} = L \\
        B' = (B^\text{used} - \{\rep{x}\}) \\
      \end{gathered}
    }{
      E[\Gamma] \vdash \lam{\rep{x{:}T}}{t} ~|~ B'
    } \breakrule
  \text{borrow-abs-fix:}~
    \dfrac{
      \begin{gathered}
        \text{$t$ is a fixpoint} \\
        E[\Gamma;\rep{\lassum{x^\varnothing}{T}}] \vdash t ~|~ B \\
        \forall z \in \{\rep{x}\}~ \text{$z$ is not linear} \\
      \end{gathered}
    }{
      E[\Gamma] \vdash \lam{\rep{x{:}T}}{t} ~|~ B
    } \breakrule
  \text{borrow-fix-fun:}~
    \dfrac{
      \begin{gathered}
        \rep{\underline{E[\Gamma;\rep{\lassum{f^\varnothing}{T}}]} \vdash t ~|~ B} \\
      \end{gathered}
    }{
      E[\Gamma] \vdash \ofix{\rep{f{:}T:=t}}{f_j} ~|~ {\tbigcup_i B_i}
    }
\end{gather*}

\begin{align*}
  &\APP(E, \Gamma, B^\text{func}, \rep{x}, T, L, B^\text{used}, B^\text{result}) \\
  &= 1 \leq |x| \\
  &~\wedge \forall i, \forall j, (i \neq j \rightarrow \neg (x_i = x_j \wedge \text{$x_i$ is linear})) \\
  &~\wedge L = \{ z | z \in \{\rep{x}\} ~\wedge~ \text{$z$ is linear} \} \\
  &~\wedge B^\text{used} = B^\text{func} \cup \tbigcup_i (\Bop_\Gamma x_i) \\
  &~\wedge B^\text{result} = B^\text{used} \cap \components_E(T) \\
  &~\wedge B^\text{used} \cap L = \varnothing \\
\end{align*}

We use a function $\components_E(T)$ to obtain the component types of a type $T$ under the global environment $E$.
It returns a set of types or $\top$.
$\top$ is a set which contains all types.
$\components_E(T)$ is defined as the minimum set which satisfy following equations.

\begin{align*}
  & \components_E(I\:\rep{t}) = \{I\:\rep{t}\} \cup \tbigcup_{(x{}:T) \in \Gamma_B} \components_E(T) \\
  & \begin{aligned}
    \quad \text{where}~
    & E[] \vdash I\:\rep{t} : S \\
    & \text{$S$ is a sort} \\
    & \text{\sf Ind}\:[p]\:(\Gamma_I := \Gamma_C) \in E \\
    & \text{$p$ the is number of recursively uniform parameters of $\text{\sf Ind}\:[p]\:(\Gamma_I := \Gamma_C)$} \\
    & I \in \Gamma_I \\
    & |t| = p \\
    & (C:\forall \Gamma_P, \forall \Gamma_A, I\:\rep{u}) \in \Gamma_C \\
    & \subst{(I\:\rep{u})}{\Gamma_P}{\rep{t}} = I\:\rep{t} \\
    & \Gamma_B = \subst{\Gamma_A}{\Gamma_P}{\rep{t}}
    \end{aligned} \\
  & \components_E(\forall T, U) = \top \\
  & \components_E(S) = \top \\
  & \quad \text{where}~ \text{$S$ is a sort} \\
\end{align*}

We extend substitution for local contexts here.
$\subst{t}{\Gamma}{\rep{u}}$ represents the term $t$ which variables in $\Gamma$ are substituted with terms $\rep{u}$.
$\subst{\Gamma'}{\Gamma}{\rep{u}}$ represents the local context $\Gamma'$ which variables in $\Gamma$ are substituted with terms $\rep{u}$.
If $\Gamma$ contains a local definition, its variable is substituted with the corresponding definition.
($\epsilon$ is the empty local context. $\varepsilon$ is the empty list of terms.)

\begin{align*}
  \subst{t}{\epsilon}{\varepsilon} &= t \\
  \subst{t}{(\lassum{x}{T};\Gamma)}{(a\:\rep{u})} &= \subst{\subst{t}{x}{a}}{\subst{\Gamma}{x}{a}}{\rep{u}} \\
  \subst{t}{(\ldef{x}{a}{T};\Gamma)}{\rep{u}} &= \subst{\subst{t}{x}{a}}{\subst{\Gamma}{x}{a}}{\rep{u}} \\
  \subst{\epsilon}{\Gamma}{\rep{u}} &= \epsilon \\
  \subst{(\lassum{x}{T};\Gamma')}{\Gamma}{\rep{u}} &= \lassum{x}{\subst{T}{\Gamma}{\rep{u}}};\subst{\Gamma'}{\Gamma}{\rep{u}} \\
  \subst{(\ldef{x}{a}{T};\Gamma')}{\Gamma}{\rep{u}} &= \ldef{x}{\subst{a}{\Gamma}{\rep{u}}}{\subst{T}{\Gamma}{\rep{u}}};\subst{\Gamma'}{\Gamma}{\rep{u}}
\end{align*}

We mix borrow information and set of variables in set-operations.
Assume $L=\{x_1,\ldots,x_n\}$ and $B=\{(T_1,y_1),\ldots,(T_m,y_m)\}$.

\begin{align*}
  B \cap L = L \cap B &= \{(T_i,y_i) \in B ~|~ 1\leq i\leq m,~ y_i \in L \} \\
  B - L &= \{(T_i,y_i) \in B ~|~ 1\leq i\leq m,~ y_i \not\in L \}
\end{align*}

We also mix borrow information and set of types (including $\top$) in set-operations.

\begin{align*}
  B \cap D = D \cap B &= \{(T_i,y_i) \in B ~|~ 1\leq i\leq m,~ T_i \in D \} & D = \{U_1, \ldots, U_n\} \\
  B \cap \top = \top \cap B &= B
\end{align*}

{\small Note:
\begin{itemize}
  \item borrow-fix-fun annotates $f_1^\varnothing \ldots f_n^\varnothing$.
    This is not correct because invoking $f_1\ldots f_n$ may refer borrowed values via free variables in
    $\ofix{\rep{f{:}T:=t}}{f_j}$.
    However, it is harmless because corresponding linear value cannot be consumed in the fix-term.
\end{itemize}}

\subsection{C Variable Allocation}\label{sec:cvaralloc}

We rename variables to be unique and approproate for C.

Since \gallina{} variables are represented by de Bruijn's indexes,
we only need to change variable names in binders:
(1) variable of abstraction,
(2) functions of fixpoint,
(3) variable of let-in, and
(4) variables of constructor members of conditional.

\begin{align*}
  \tD &= \lamT{\fbox{$x$}}{T}{\tD'} & \leftarrow (1) \\
      &\bnfor \ofix{\rep{\fbox{$f$}/k{:}T:=\tD}}{f_j} & \leftarrow (2) \\
  \tD' &= \tD \\
       &\bnfor \tL & \text{the type of $\tL$ is an inductive type} \\
  \tL &= \letinM{\fbox{$x$}}{\tM:T}{\tM} & \leftarrow (3) \\
  \tM &= \omatch{x}{\rep{C\:\rep{\fbox{$x$}}\Rightarrow \tL}} & \leftarrow (4) \\
      &\bnfor \tE \\
  \tE &= x \\
    &\bnfor c \bnfor C & \text{The type of $c$ and $C$ are inductive type} \\
    &\bnfor \tF\:\rep{x} & \text{$0 < |x|$, the type of $\tF\:\rep{x}$ is inductive type}  \\
    &\bnfor \tD \\
  \tF &= x \bnfor c \bnfor C \bnfor \ofix{\rep{\fbox{$f$}/k{:}T:=\tD}}{f_j} & \leftarrow (2)
\end{align*}

\section{C Code Generation}\label{sec:c-code-gen}
\subsection{The \gallina{} Subset For C Code Generation}\label{sec:gallinasubsetforcgen}

\begin{align*}
  \tD &= \lamT{x}{T}{\tD'} \\
      &\bnfor \ofix{\rep{f/k{:}T:=\tD}}{f_j} \\
  \tD' &= \tD \\
       &\bnfor \tL & \text{the type of $\tL$ is an inductive type} \\
  \tL &= \letinM{x}{\tM:T}{\tM} \\
  \tM &= \omatch{x}{\rep{C\:\rep{x}\Rightarrow \tL}} \\
      &\bnfor \tE \\
  \tE &= x \\
    &\bnfor c \bnfor C & \text{The type of $c$ and $C$ are inductive type} \\
    &\bnfor \tF\:\rep{x} & \text{$0 < |x|$, the type of $\tF\:\rep{x}$ is inductive type}  \\
    &\bnfor \tD \\
  \tF &= x \bnfor c \bnfor C \bnfor \ofix{\rep{f/k{:}T:=\tD}}{f_j}
\end{align*}

\subsection{Detection of Inlinable Fixpoints}\label{sec:inlinable-fixpoint-detection}
We detect inlinable fixpoints.
``Inlinable fixpoint'' means a fixpoint, $\ofix{\rep{f:=t}}{f_j}$,
which all application to $f_i$ is located at the tail positions of $\rep{f}$.
In this case, the continuation of the applications to $\rep{f}$
in $\letin{x}{(\ofix{\rep{f:=t}}{f_j})\;\rep{y}}{u}$
are always $\letin{x}{\Box}{u}$.
Thus, we can translate the tail positions of $\rep{f}$ to
(1) assignments to the arguments of $f_i$ and $\kwgoto{}\;f_i$ for application of $f_i$ and
(2) assignment to $x$ and $\kwgoto{}\;u$ otherwise.
This translation is equivalent to inlining a tail-recursive function, which means generating a loop at a non-tail position.

$\RNT{t} = (\TR{t}, \N{t}, \T{t})$ classify variables in $t$.
\begin{align*}
  \TR{t} &: \text{tail-recursive fixpoint bounded functions that do not need to be real functions in $t$} \\
  \N{t} &: \text{free variables at non-tail positions of $t$} \\
  \T{t} &: \text{free variables at tail positions of $t$}
\end{align*}
``tail position'' is extended to the function position of the application at a tail position. \\
$\TRop$ distinguishes fixpoint bounded functions translatable without actual functions (but with \kwgoto{}) or not.
\begin{align*}
  \RNT{x\:\rep{y}} &= (\varnothing, \{\rep{y}\}, \{x\}) \\
  \RNT{c\:\rep{y}} &= (\varnothing, \{\rep{y}\}, \varnothing) \\
  \RNT{C\:\rep{y}} &= (\varnothing, \{\rep{y}\}, \varnothing) \\
  \RNT{\letin{x}{t}{u}} &= (\TR{t} \cup \TR{u}, \N{t} \cup \T{t} \cup \N{u} - \{x\}, \T{u} - \{x\}) \\
  \RNT{\omatch{y}{\rep{C\:\rep{x}\Rightarrow t}}} &= \bigl(\tbigcup_i \TR{t_i}, \tbigcup_i \N{t_i} - \{\rep{x_i}\}, \tbigcup_i \T{t_i} - \{\rep{x_i}\} \bigr) \\
  \RNT{\lam{x}{t}} &= (\TR{t}, (\N{t} \cup \T{t}) - \{x\}, \varnothing) \\
  \RNT{(\ofix{\rep{f:=t}}{f_j})\:\rep{y}} &=
    \begin{cases}
      \begin{aligned}[t]
        \bigl(& \tbigcup_i \TR{t_i} \cup \{\rep{f}\}, \\
              & \bigl(\tbigcup_i \N{t_i} - \{\rep{f}\}\bigr) \cup \{\rep{y}\}, \\
              & \tbigcup_i \T{t_i} - \{\rep{f}\} \bigr)
      \end{aligned}
        & (0 < |y|) \wedge \bigl(\tbigcup_i \N{t_i} \cap \{\rep{f}\} = \varnothing\bigr) \\
      \begin{aligned}[t]
        \bigl(& \tbigcup_i \TR{t_i}, \\
              & \bigl(\tbigcup_i (\N{t_i} \cup \T{t_i}) - \{\rep{f}\}\bigr) \cup \{\rep{y}\}, \\
              & \varnothing \bigr)
      \end{aligned}
        & \text{otherwise}
    \end{cases}
\end{align*}
{\small Note:
\begin{itemize}
  \item The variables in $t$ are unique.
    \codegen{} uses de Bruijn's indexes for $\Nop$ and $\Top$;
    the variables renamed by \secref{sec:cvaralloc} for $\TRop$.
  \item $y$ of $\omatch{y}{\rep{C\:\rep{x}\Rightarrow t}}$ is not counted because $y$ is not a function and does not affect the final $\TRop$.
\end{itemize}}

\subsection{Top-Level Functions Detection}\label{sec:top-level-function-detection}
If a fixpoint needs recursive call in C, we need a real C function for it.
\codegen{} detects such fixpoints by simulating $A_K$ and $B_K$ in \secref{sec:AK} and \secref{sec:BK} to collect application of fixpoint-bounded functions.

\subsection{Fix-lifting}\label{sec:fix-lifting}

We use lambda-lifting-like technique to translate fixpoint expressions without closures.

Consider following (artificial) example.

\begin{lstlisting}
Definition c x y :=
  fix f n :=
  match n with
  | O => x                  (* invocation of f needs x *)
  | S n' =>
    (fix g m :=
    match m with
    | O => y + f n'         (* invocation of g needs y and f *)
    | S m' => S (g m')
    end) n
  end.
\end{lstlisting}

\codegen{} translate a \gallina{} application to C function call (if \kwgoto{} is not usable).
In this scenario, \codegen{} generates C functions corresponding to internal functions \lstinline[style=Cstyle]!f! and \lstinline[style=Cstyle]!g!.
The application, \lstinline!g m'!, is translated to \lstinline[style=Cstyle]!g(m')!.
But it does not work because \lstinline!g! needs \lstinline!y!.
Also, \lstinline!g! needs \lstinline!f!.
Although there is the C function \lstinline!f!, \lstinline!f! needs \lstinline!x!.
Thus, \codegen{} need to add extra arguments as \lstinline[style=Cstyle]!g(x, y, m')! which is similar to lambda-lifting.
We call this translation, adding extra arguments for functions bounded by fixpoints, fix-lifting.

We define $\FIXFUNCS$, $\FIXFV$, $\EXARGS'$, and $\EXARGS$ for each definition \kwDefinition~$c := u$.

$\EXARGS(f)$ is the extra arguments for the fix-bounded function $f$ in the definition.

$\FIXFUNCS$ is the set of fix-bounded functions in $u$.

\[
  \FIXFUNCS = \bigcup_{K\left[\ofix{\rep{f:=t}}{f_j}\right] = u} \{\rep{f}\}
\]

$\FIXK(x)$ is the context of the fixpoint which bounds $x$.
It is $\bot$ if $x$ is not a function bounded by a fixpoint.

\[
  \FIXK(x) =
  \begin{cases}
    K & u = K[\ofix{\repi{(f_l:=t_l)}{1\leq l < i} (x:=t_i) \repi{(f_l:=t_l)}{i < l \leq h}}{f_j}]  \\
    \bot & \text{otherwise}
  \end{cases}
\]

$\FIXFV(x)$ is the set of free variables of the fixpoint which bounds $x$.
If $x$ is not a fix-bounded function, $\FIXFV(x)=\varnothing$.

\[
  \FIXFV(x) =
  \begin{cases}
    \FV(t) & u = \FIXK(x)[t]  \\
    \varnothing & \text{otherwise}
  \end{cases}
\]

$\EXARGS'(x)$ is a set which satisfy the following conditions.
$\EXARGS'(f)$ is similar to $\FIXFV(f)$ which is the set of free variables of the fixpoint which bounds $f$.
But if it contains a function bounded by an outer fixpoint, the free variable of the outer fixpoint are also contained.

\begin{align*}
  \EXARGS'(x) \supseteq& \FIXFV(x) \cup \bigcup_{y \in \FIXFV(x)} \EXARGS'(y) \\
  \EXARGS'(x) \subseteq& \KV(\FIXK(x))
\end{align*}

\codegen{} chooses the minimal set for $\EXARGS'(f)$ if a dedicated internal C function is generated for $f$.
But if $f$ is callable via $c$ like follows, \codegen{} generate a call to $c$ for $f$
to avoid generating the dedicated C function for \texttt{f}.
This means $\texttt{f}\:\texttt{y'}$ is translated to \lstinline[style=Cstyle]!c(x, y')!.
In this case, the arguments of the external C function for $c$ must corresponds to the type of $c$.
Thus \codegen{} chooses the maximal set (all bound variables) for $\EXARGS'(f)$.
(This means that \texttt{x} is passed even if \texttt{f} does not use \texttt{x}.)

\[
  \kwDefinition~\texttt{c} := \lam{\texttt{x}}{\ofix{\texttt{f}:=\lam{\texttt{y}}{\ldots \texttt{f}\:\texttt{y'} \ldots}}{\texttt{f}}}
\]

$\EXARGS(x)$ is defined as follows.
It is $\EXARGS'(x)$ except fix-bounded functions.

\[
  \EXARGS(x) = \EXARGS'(x) - \FIXFUNCS
\]

When $\EXARGS(x)$ is used in a context which the order matters,
we consider it is a list of variables from declared ouside to inside.
(used in \secref{sec:AK})

When $\EXARGS(x)$ is used in a context which require types,
we consider it is a set of pairs of variable and its type.
(used in \secref{sec:genfunm})

\subsection{Translation to C for a Non-Tail Position}\label{sec:AK}
$\A{K}{t}$ generates C code for $t$ in a non-tail position.
The result expression is passed to $K$. \\
$K(e) = \dq{v = e\ttsemi}$ in simple situations.
\begin{align*}
  &\A{K}{x} = K(\dq{x}) \\
  &\A{K}{f\:\rep{x}} =
      \begin{aligned}[t] \ldq & {\passign(\fvarsd{f}, \rep{x})} \\ & \kwgoto\:\mathtt{entry\_}f\ttsemi \rdq \end{aligned}
    && \begin{aligned}[t] & (|x| > 0) \wedge \text{$f$ is bounded by a fixpoint} \wedge {} \\ & f \in \tr \end{aligned} \\
  &\A{K}{f\:\rep{x}} = K(\dq{f\ttparen{\rep{y}\ttcomma \rep{x}}})
    && \begin{aligned}[t] & (|x| > 0) \wedge \text{$f$ is bounded by a fixpoint} \wedge {} \\ & f \not\in \tr \quad \text{where} \quad \rep{y} = \EXARGS(f) \end{aligned} \\
      &\A{K}{c\:\rep{x}} = K(\dq{c\ttparen{\rep{x}}})                                   && |x| \geq 0 \\
  &\A{K}{C\:\rep{x}} = K(\dq{C\ttparen{\rep{x}}})                                   && |x| \geq 0 \\
  &\A{K}{\letin{x}{t_1}{t_2}} = \ldq \A{K'}{t_1}\; \A{K}{t_2} \rdq
    && \text{where}\quad K'(e) = \dq{x\:\tteq\:e\ttsemi} \\
  &\A{K}{\omatch{x}{\rep{C\:\rep{y}\Rightarrow t}}} && \text{where}\quad x : T \\
     & \quad\begin{alignedat}{2}
       \ldq & \kwswitch\:\ttparen{\mathit{swfunc}_T\ttparen{x}}\:\ttlbrace \\
            & \cdots \\
            & \mathit{caselabel}_{C_i}\ttcolon\quad
              \begin{aligned}[t]
              & \repi{y_{ij}\:\tteq\:\mathit{get\_member}_{C_i j}(x)\ttsemi}{j} \\
              & \mathit{linear\_dealloc}_{T}(x)\ttsemi \\
              & {\A{K}{t_i}} \\
              & \kwbreak\ttsemi
              \end{aligned} \\
            & \cdots \\
            & \ttrbrace\rdq
       \end{alignedat} \\
  &\A{K}{(\ofix{\rep{f:=t}}{f_j})\:\rep{x}} =                   && f_j \in \tr \\
     & \quad\begin{alignedat}[t]{2}
       \ldq & \passign(\fvars{t_j}, \rep{x}) \\
            & {\genbodyat{K'}{\ofix{\rep{f:=t}}{f_j}}} \\
            & \mathtt{exit\_}f_j\ttcolon \rdq
       \end{alignedat} &&
               \begin{alignedat}[t]{1}
                  & \text{where} \\
                  & K'(e) =
                    \begin{cases}
                    K(e) & \text{$K(e)$ contains \kwgoto}  \\
                    \begin{aligned}[t]
                      \ldq & K(e) \\
                           & \kwgoto\:\mathtt{exit\_}f_j\ttsemi \rdq
                    \end{aligned} & \text{otherwise}
                  \end{cases}
                \end{alignedat} \\
  &\A{K}{(\ofix{\rep{f:=t}}{f_j})\:\rep{x}} =                      && f_j \not\in \tr \\
     & \quad\begin{alignedat}[t]{2}
       \ldq & K(f_j\ttparen{\rep{y}\ttcomma \rep{x}})                  \\
            & \kwgoto\:\mathtt{skip\_}f_j\ttsemi                                    \\
            & {\genbodyan{\ofix{\rep{f:=t}}{f_j}}}                          \\
            & \mathtt{skip\_}f_j\ttcolon \rdq
       \end{alignedat} &&
               \begin{alignedat}[t]{1}
                  & \text{where} \\
                  & \rep{y} = \EXARGS(f_j)
               \end{alignedat}
\end{align*}
{\small Note:
\begin{itemize}
  \item $\dq{\cdots}$ means a string.
    A string can contain characters in typewriter font and expressions starting in italic or roman font.
    The former is preserved as-is.
    The latter embeds the value of the expression (with name translation from \gallina{} to C).
  \item \gallina{} types, constants, and constructors have corresponding (user-configurable) C names and they are implicitly translated.
    \gallina{} variables are translated by the mapping defined in \secref{sec:cvaralloc}.
  \item $\tr = \TR{t}$ where the translating function is defined as \kwDefinition~$c := t$.
  \item $\mathit{swfunc}_T, \mathit{caselabel}_{C_i}$, and $\mathit{get\_member}_{C_i j}$ are defined by a user to translate \kwmatch-expressions for the inductive type $T$.
  \item $\mathit{linear\_dealloc}_{T}(x)$ is the deallocation function for the linear type $T$.  It is empty for unrestricted types.
  \item $\passign(\rep{y}, \rep{x})$ is a parallel assignment. It is translated to a sequence of assignments to assign $x_1\ldots x_n$ into $y_1\ldots y_n$.  It may require temporary variables.
  \item We do not define $\A{K}{\lam{x}{t}}$ because we do not support closures yet.
  \item Actual \codegen{} generates $\genbodyan{}$ in a different position to avoid the label $\mathtt{skip\_}f_j$ and $\kwgoto\:\mathtt{skip\_}f_j\ttsemi$.
\end{itemize}}

\subsection{Translation to C for a Tail Position}\label{sec:BK}
$\B{K}{t}$ generates C code for $t$ in a tail position.
The result expression is passed to $K$. \\
$K(e) = \dq{\kwCreturn\:e\ttsemi}$ in simple situations.
\begin{align*}
  &\B{K}{x} = K(\dq{x}) \\
  &\B{K}{f\:\rep{x}} = \begin{aligned}[t] \ldq & {\passign(\fvarsd{f}, \rep{x})} \\ & \kwgoto\:\mathtt{entry\_}f\ttsemi \rdq \end{aligned}
    && (|x| > 0) \wedge \text{$f$ is bounded by a fixpoint} \\
  &\B{K}{c\:\rep{x}} = K(\dq{c\ttparen{\rep{x}}})                                   && |x| \geq 0 \\
  &\B{K}{C\:\rep{x}} = K(\dq{C\ttparen{\rep{x}}})                                   && |x| \geq 0 \\
  &\B{K}{\letin{x}{t_1}{t_2}} = \ldq \A{K'}{t_1}\; \B{K}{t_2} \rdq
    && \text{where}\quad K'(e) = \dq{x\:\tteq\:e\ttsemi} \\
  &\B{K}{\omatch{x}{\rep{C\:\rep{y}\Rightarrow t}}} = && \text{where}\quad x : T \\
     & \quad\begin{alignedat}{2}
       \ldq & \kwswitch\:\ttparen{\mathit{swfunc}_T\ttparen{x}}\:\ttlbrace \\
            & \cdots \\
            & \mathit{caselabel}_{C_i}\ttcolon\quad
              \begin{aligned}[t]
                & \repi{y_{ij}\:\tteq\:\mathit{get\_member}_{C_i j}(x)\ttsemi}{j} \\
                & \mathit{linear\_dealloc}_{T}(x)\ttsemi \\
                & {\B{K}{t_i}}
              \end{aligned} \\
            & \cdots \\
            & \ttrbrace\rdq
     \end{alignedat} \\
  &\B{K}{(\ofix{\rep{f:=t}}{f_j})\:\rep{x}} =       \\
     & \quad\begin{alignedat}{2}
       \ldq & {\passign(\fvars{t_j}, \rep{x})} \\
            & {\genbodyb{K}{\ofix{\rep{f:=t}}{f_j}}} \rdq
       \end{alignedat}
\end{align*}
{\small Note:
\begin{itemize}
  \item We do not define $\B{K}{\lam{x}{t}}$ because a tail position cannot be a function after the argument completion.
\end{itemize}}

\subsection{Auxiliary Functions for Translation to C}\label{sec:aux-function}
\[
  \fvars{t} =
  \begin{cases}
    \dq{x;\:\fvars{u}} & t = \lam{x}{u} \\
    \fvars{t_j}       & t = \ofix{\rep{f:=t}}{f_j} \\
    \dq{}             & \text{otherwise}
  \end{cases}
\]
\[
  \fvarsd{f_i} = \fvars{t_i} \quad \text{for functions bounded by $\ofix{\rep{f:=t}}{f_j}$}
\]
\[
  \genbodyat{K}{t} =
  \begin{cases}
    \genbodyat{K}{u}                            & t = \lam{x}{u} \\
    \dq{\mathtt{entry\_}f_i\ttcolon\:\genbodyat{K}{t_i}} & t = \ofix{\rep{f:=t}}{f_j} \\
    \quad \text{for}~i=j, 1,\dotsc, (j-1), (j+1),\dotsc, |f| \\
    \A{K}{t}                            & \text{otherwise}
  \end{cases}
\]
\[
  \genbodyan{t} =
  \begin{cases}
    \genbodyan{u}                            & t = \lam{x}{u} \\
    \dq{\mathtt{entry\_}f_i\ttcolon\:\genbodyan{t_i}} & t = \ofix{\rep{f:=t}}{f_j} \\
    \quad \text{for}~i=1,\dotsc, |f| \\
    \B{K}{t}                            & \begin{aligned}[t]
                                            & \text{otherwise} \\
                                            & \text{where} \\
                                            & \quad t:T \\
                                            & \quad K(e) = \dq{\texttt{*($T$*)ret = $e$; \kwCreturn;}}
                                            \end{aligned}
  \end{cases}
\]
\[
  \genbodyb{K}{t} =
  \begin{cases}
    \genbodyb{K}{u}                            & t = \lam{x}{u} \\
    \dq{\mathtt{entry\_}f_i\ttcolon\:\genbodyb{K}{t_i}} & t = \ofix{\rep{f:=t}}{f_j} \\
    \quad \text{for}~i=j, 1,\dotsc, (j-1), (j+1),\dotsc, |f| \\
    \B{K}{t}                            & \text{otherwise}
  \end{cases}
\]
{\small Note:
\begin{itemize}
  \item $\fvarsop$ and $\fvarsop'$ returns a list of variables: $x_1;\ldots;x_n;$.  For simplicity, we omit ``$;$'' if not ambiguous.
  \item $\dq{g(i)}~\text{for}~i=j_1,\dotsc,j_n$ means $\dq{g(j_1)\:\ldots\:g(j_n)}$.
\end{itemize}}


\subsection{Translation for a Top-Level Function which is Translated to Multiple C Functions}\label{sec:genfunm}
$\genfunm{c}$ translates the function (constant) $c$ with one or more auxiliary functions.
We assume $c$ is defined as \kwDefinition~$c := t.$
The auxiliary functions $f_1 \ldots f_n$ are fixpoint bounded functions in $t$ which are invoked as functions.
We assume the types of them:
\begin{align*}
  c &: T_{01} \rightarrow \dotsb \rightarrow T_{0m_0} \rightarrow T_{00} \\
  f_i &: T_{i1} \rightarrow \dotsb \rightarrow T_{im_i} \rightarrow T_{i0} && i = 1\ldots n
\end{align*}
\[ \text{where} \quad \text{$T_{i0}$ are inductive types ($i=0\ldots n$)} \]
The formal arguments of $c$ are $x_{01}\ldots x_{0m_0} = \fvars{t}$ and
the formal arguments of $f_i$ are $x_{i1}\ldots x_{im_i} = \fvarsd{f_i}$.

$f_i$ invocation in C needs extra arguments, $\EXARGS(f_i) = y_{i1}\mathord{:}U_{i1} \ldots y_{il_i}\mathord{:}U_{il_i}$, addition to the actual arguments in \gallina{} application because the free variables of the fixpoint should also be passed.
If the free variables contain a function bounded by an outer fixpoint, the function itself is not passed but the free variables of the outer fixpoint are also passed.
We iterate it until no fixpoint functions.

\begin{alignat*}{2}
  \genfunm{c} &= \ldq && \enumentries{c}~\argstructdefs{c}~\forwarddecl{c}~\entryfunctions{c}~\bodyfunction{c} \rdq
\end{alignat*}
\[ \enumentries{c} = \dq{\kwenum\:\mathtt{enum\_func\_}c\:\ttbrace{ \mathtt{func\_}c \repi{\ttcomma \mathtt{func\_}f_i}{i=1\ldots n} }\ttsemi} \]
\[ \argstructdefs{c} = \dq{\mainstructdef{c}\:\repi{\auxstructdef{c}{i}}{i=1\ldots n} } \]
\[ \mainstructdef{c} = \dq{\kwstruct\:\mathtt{arg\_}c\:\ttbrace{\:\repi{T_{0j}\:\mathtt{arg}j\ttsemi}{j=1\ldots m_0} \:}\ttsemi} \]
\[ \auxstructdef{c}{i} = \dq{\kwstruct\:\mathtt{arg\_}f_i\:\ttbrace{\:\repi{U_{ij}\:\mathtt{exarg}j\ttsemi}{j=1\ldots l_i}\:\repi{T_{ij}\:\mathtt{arg}j\ttsemi}{j=1\ldots m_i}\:}\ttsemi} \]
\[ \forwarddecl{c} = \dq{\kwstatic\:\kwvoid\:\mathtt{body\_function\_}c\ttparen{\kwenum\:\mathtt{enum\_func\_}c\:\mathtt{g}\ttcomma \kwvoid\:\mathtt{\ttstar arg}\ttcomma \kwvoid\:\mathtt{\ttstar ret}}\ttsemi} \]
\[ \entryfunctions{c} = \dq{\mainfunction{c}\:\repi{\auxfunction{c}{i}}{i=1\ldots n}} \]
\begin{alignat*}{2}
  \mainfunction{c} &= \ldq && \kwstatic\:T_{00}\:c\ttparen{\repopi{T_{0j}\:x_{0j}}{\ttcomma}{j=1\ldots m_0}}\:\ttlbrace \\
  & && \quad \kwstruct\:\mathtt{arg\_}c\:\mathtt{arg}\:\tteq\:\ttbrace{ \repopi{x_{0i}}{\ttcomma}{i=1\ldots m_0} }\ttsemi\:T_{00}\:\mathtt{ret}\ttsemi \\
  & && \quad \mathtt{body\_function\_}c\ttparen{\mathtt{func\_}c\ttcomma \ttamp\mathtt{arg}\ttcomma \ttamp\mathtt{ret}}\ttsemi \kwCreturn\:\mathtt{ret}\ttsemi \\
  & && \ttrbrace \rdq
\end{alignat*}
\begin{alignat*}{2}
  \auxfunction{c}{i} &= \ldq && \kwstatic\:T_{i0}\:f_i\ttparen{\repopi{U_{ij}\:y_{ij}}{\ttcomma}{j=1\ldots l_i}\; \repopi{T_{ij}\:x_{ij}}{\ttcomma}{j=1\ldots m_i}}\:\ttlbrace \\
  & && \quad \kwstruct\:\mathtt{arg\_}f_i\:\mathtt{arg}\:\tteq\:\ttbrace{ \repopi{y_{ij}}{\ttcomma}{j=1\ldots l_i}\; \repopi{x_{ij}}{\ttcomma}{j=1\ldots m_i} }\ttsemi\:T_{i0}\:\mathtt{ret}\ttsemi\\
  & && \quad \mathtt{body\_function\_}c\ttparen{\mathtt{func\_}f_i\ttcomma \ttamp\mathtt{arg}\ttcomma \ttamp\mathtt{ret}}\ttsemi\:\kwCreturn\:\mathtt{ret}\ttsemi \\
  & && \ttrbrace \rdq
\end{alignat*}
\begin{alignat*}{2}
  \bodyfunction{c} &= \ldq
    && \kwstatic\:\kwvoid\:\mathtt{body\_function\_}c\ttparen{\kwenum\:\mathtt{enum\_func\_}c\:\mathtt{g}\ttcomma \kwvoid\:\ttstar\mathtt{arg}\ttcomma \kwvoid\:\ttstar\mathtt{ret}}\:\ttlbrace \\
  & && \quad \mathit{decls} \\
  & && \quad \kwswitch\:\ttparen{\mathtt{g}}\:\ttbrace{\:\repi{\auxcase{c}{i}}{i=1\ldots n} \:\maincase{c}\:} \\
  & && \quad {\genbodyb{K}{t}} \\
  & && \ttrbrace \rdq
\end{alignat*}
\begin{align*}
  \auxcase{c}{i} = \ldq & \kwcase\:\mathtt{func\_}f_i\ttcolon \\
  & \repi{y_{ij}\:\tteq\:\ttparen{\ttparen{\kwstruct\:\mathtt{arg\_}f_i\:\ttstar}\mathtt{arg}}\texttt{->}\mathtt{exarg}j\ttsemi}{j=1\ldots l_i} \\
  & \repi{x_{ij}\:\tteq\:\ttparen{\ttparen{\kwstruct\:\mathtt{arg\_}f_i\:\ttstar}\mathtt{arg}}\texttt{->}\mathtt{arg}j\ttsemi}{j=1\ldots m_i} \\
  & \kwgoto\:\mathtt{entry\_}f_i\ttsemi \rdq
\end{align*}
\begin{align*}
  \maincase{c} = \ldq & \kwdefault\ttcolon\ttsemi \\
  & \repi{x_{0j}\:\tteq\:\ttparen{\ttparen{\kwstruct\:\mathtt{arg\_}c\:\ttstar}\mathtt{arg}}\texttt{->}\mathtt{arg}j\ttsemi}{j=1\ldots m_0} \rdq
\end{align*}
\[ \text{where} \quad
  \begin{aligned}[t]
    & \text{$\mathit{decls}$ is local variable declarations for variables used in $\genbodyb{K}{t}$.} \\
    & K(e) = \dq{\ttstar\ttparen{T_{00}\ttstar}\texttt{ret}\:\tteq\:e\ttsemi \kwCreturn\ttsemi}
  \end{aligned}
\]

\subsection{Translation for a Top-Level Function which is Translated to a Single C Function}\label{sec:genfuns}
$\genfuns{c}$ translates the function (constant) $c$ to a single C function.
\[
  \genfuns{c} = \dq{\kwstatic\:T_0\:c\ttparen{\fargsd{t}}\:\ttbrace{\:\mathit{decls}\:\genbodyb{K}{t}\:}}
\]
\[ \text{where} \quad
  \begin{aligned}[t]
    & \text{$c$ is defined as \kwDefinition~$c : T_1 \rightarrow \dotsb \rightarrow T_n \rightarrow T_0 := t.$} \\
    & \text{$T_0$ is an inductive type} \\
    & \text{$\mathit{decls}$ is local variable declarations for variables used in $\genbodyb{K}{t}$ excluding $\fargs{t}$.} \\
    & K(e) = \dq{\kwCreturn\:e\ttsemi} \\
    & \fargs{t} =
      \begin{cases}
        \dq{T\:x\ttcomma\:\fargs{u}}      & t = \lamT{x}{T}{u} \\
        \fargs{t_j}       & t = \ofix{\rep{f:=t}}{f_j} \\
        \dq{}    & \text{otherwise}
      \end{cases} \\
    & \fargsd{t} = \fargs{t}~\text{without the trailing comma}
  \end{aligned}
\]

\subsection{Translation for Top-Level Function}\label{sec:genfun}
\[
  \genfun{c} =
  \begin{cases}
    \genfunm{c} & \text{$t$ needs multiple functions} \\
    \genfuns{c} & \text{otherwise}
  \end{cases}
\]
\[ \text{where} \quad
  \begin{aligned}[t]
    & \text{$c$ is defined as \kwDefinition~$c := t.$}
  \end{aligned}
\]

\section{Verification of \gallina-to-\gallina{} Transformations}\label{sec:verification-of-gallina-to-gallina-transformations}

\codegen{} verifies each step of the \gallina-to-\gallina{} transformations.

Since all transformations except match-app are convertible,
\codegen{} checks convertibility of such transformations.

  %&\B{K}{\omatch{x}{\rep{C\:\rep{y}\Rightarrow t}}} = && \text{where}\quad x : T \\

Assuming $t = (\omatch{x}{\rep{C\: \rep{y}\Rightarrow t}}\: z)$,
$u = (\omatch{x}{\rep{C\: \rep{y} \Rightarrow t\: \underline{z}}})$, and
some context $K$,
match-app transformation is $K[t] \reltri K[u]$.
The match-app redex, $t$, is a subterm of a whole function, $K[t]$.
Free variables in $t$ and $u$ are bounded by $K$.

\codegen{} verifies match-app transformation by proving $t = u$ internally.
The proof term is not visible from a user.

\codegen{} does not verify $K[t] = K[u]$ because
(1) this verification does not add big guarantee over above because we know same context $K$ is used in LHS and RHS,
(2) induction is required if $x$ is a decreasing argument of fixpoint in $K$ and it is difficult to automate.

\bibliographystyle{plain}
\bibliography{base}

\end{document}
\endinput
